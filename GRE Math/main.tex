\documentclass{ximera}
\usepackage{amsmath}


\newcommand*{\Perm}[2]{{}^{#1}\!P_{#2}}%
\newcommand*{\Comb}[2]{{}^{#1}C_{#2}}%

\title{ALES \& GRE Notlari - Erhan Tezcan}
\author{Erhan Tezcan}

\begin{document}
\begin{abstract}
    ALES ve GRE notlarim bu dokumandadir.
\end{abstract}
\maketitle

\section{Hos Sorular}
Bu bolumde genel olarak hosuma giden sorular yer aliyor.
\begin{example}
$$
\frac{9}{16} < \frac{6}{x} < \frac{4}{5}
$$
kosulunu saglayan $x$ tam sayilarinin toplami kactir?
\end{example}
\begin{explanation}
$$
\frac{16}{9} > \frac{x}{6} > \frac{5}{4}
$$
$$
\frac{16\times4}{9\times4} > \frac{x\times6}{6\times6} > \frac{5\times9}{4\times9}
$$
$$
\frac{64}{36} > \frac{6x}{36} > \frac{45}{36}
$$
$$
64 > 6X > 45
$$
$$
10,... > x > 7,...
$$
O zaman $x = 8, 9, 10$ ve toplamlari $27$.
\end{explanation}

Genel olarak inequility cozerken taraflarin isareti cok onemli! Eksi ile ugrasmak ozellikle*** \par

Standart Sapma sorularinda hesaplamaktan cok nasil dagilmislar bakilabilir, mantik kullan yani. \par 

\begin{example}
What is first number that is not a factor of $20!$? 
\end{example}
\begin{explanation}
20 ve daha onceki sayilar hepsi factor, $21 = 7\times3$ ve $22=2\times11$ ama $23$ asal dolayisiyla cevap $23$.
\end{explanation}

\begin{example}
2 evli cift ve 2 bekardan olusan 6 kisilik bir grup bir kanepeye oturacaktir, ciflerin her biri esleriyle yan yana olacaktir ve bekarlar yan yana olmayacaktir. Bu kisiler kanepeye kac farkli bicimde oturabilirler? 
\end{example}
\begin{explanation}
\begin{equation*}
    \begin{split}
        \text{Istenilen durum} &= \text{Ciftler yan yana bekarlar serbest} \\
        &- \text{Ciftler yanyana berkarlar yan yana} \\
        &= 4! \times 2! \times 2! - 3! \times 2! \times 2! \times 2! = 96
    \end{split}
\end{equation*}
\end{explanation}

\begin{example}
Bir torbada 3 beyaz 4 kirmizi top vardir. Bu torbadan ayni anda rastgele cekilen iki toptan en az birinin beyaz olma olasiligi kactir?
\end{example}
\begin{explanation}
En az birinin beyaz olmasi, toplam olasiliktan sadece kirmizi olmasinin cikarilmasi ile bulunabilir. Dolayisiyla
\[
P = 1 - \frac{4}{7}
\]
\end{explanation}

\begin{example}
$ A = \{a, b, c, d, e, f, g\}$kumesinin elemanlari kullanilarka biri 3 elemanli, digeri 4 elemanli iki ayrik kume olusturulmak isteniyor. Buna gore $a$ ve $b$ elemanlari ayni kumede bulunmayacak bicimde, bu iki kume kac farkli sekilde olusturulabilir?
\end{example}
\begin{explanation}
$a$ ve $b$ ayni kumede bulunmayacaksa $a$ ya da $b$ den birinin 3 elemanli, digerinin 4 elemanli alt kumede oldugu dusunulur. Buna gore:
$$
\binom{5}{2}\times \binom{3}{3}\times 2 \times 1 = 20
$$

\end{explanation}


\section{Permutasyon \& Kombinasyon}
\subsection{Kombinasyon}
$n$ elemanli bir kumenin $r$'li kombinasyonu sayisi $C(n,r)$ seklinde gosterilir. Kombinasyonlarda mesela $\{a,b\}$ hesaplandiysa $\{b,a\}$ hesaplanmaz, yani permutasyonun aksine siralama onemli degil.
\begin{equation}
    \binom{n}{r} = \frac{n!}{(n-r)! \times r!}
\end{equation}
Kendimce aklimda tuttugum sekil soyle:
\begin{equation}
    \binom{n}{r} = \frac{\overbrace{n\times(n-1)\times...}^{r\text{ tane}}}{r!}
\end{equation}
\begin{remark}
\begin{equation}
    \binom{n}{r} = \binom{n}{n-r}
\end{equation}
\begin{equation}
    \binom{n}{1} = n
\end{equation}
\begin{equation}
    \binom{n}{0} = 1
\end{equation}
\end{remark}

\begin{example}
8 kisi arasindan kurulacak 6 kisilik takim kac farkli sekilde secilebilir?
\end{example}
\begin{explanation}
$$
\binom{8}{6} = 28
$$
\end{explanation}

\begin{example}
Bir cember ustunde 5 nokta vardir, bu bes noktanin (a) ikisinden gecen  kac farkli dogru cizilebilir? (b) uc noktasindan kosesi olan kac farkli ucgen cizilebilir?
\end{example}
\begin{explanation}
$$
\text{(a)    } \binom{5}{2} = 10
$$
$$
\text{(b)    } \binom{5}{3} = 10
$$
\end{explanation}

\begin{example}
$A = \{1, 2, 3, 4, 5, 6\}$ kumesinin uc elemanli alt kumelerinin kac tanesinde en az 2 tane cift sayi bulunur?
\end{example}
\begin{explanation}
Iki turlu olabilir: 2 cift 1 tek ve 3 cift sayi secilmis olabilir. 3 cift ve 3 tek sayi var zaten, oyleyse
$$
\binom{3}{2}\binom{3}{1} + \binom{3}{3} = 9 + 1 = 10
$$
\end{explanation}


\begin{example}
5 kitaptan 3 tanesi secilip bir rafa dizilecektir. Kac farkli sekilde secim yapilabilir? (Simdi kombinasyon kullanilacak iste)
\end{example}
\begin{explanation}
$$
C(5,3) = 10
$$
\end{explanation}


\subsection{Permutasyon}
Permutasyonda ise dizilim (siralama) onemlidir, kombinasyonda bu onemli degildir kombinasyonda sadece secim yapilir. 

\begin{equation}
    P(n,r) = \frac{n!}{(n-r)!}
\end{equation}

\begin{example}
5 kitaptan 3 tanesi secilip bir rafa dizilecektir. Kac farkli sekilde dizilim yapilabilir?
\end{example}
\begin{explanation}
$$
P(5,3) = 60
$$
\end{explanation}

Direkt olarak $n$ tane urunun siralanmasi: 
\begin{equation}
    n!
\end{equation}
\begin{example}
5 kitaptan bir rafa kac farkli sekilde dizilebilir?
\end{example}
\begin{explanation}
$$
P(5,5) = 5! = 120
$$
\end{explanation}
\subsection{Compound Interest (Bilesik Faiz)}
Diyelim elimizde $P$ para var, yillik yuzde $f$ faizden $n$ yil duruyor. Sonunda elimizde ne kadar para olur dersek:
\begin{equation}
    P\left(1 + \frac{f}{100}\right)^y
\end{equation}
Aslinda formul ezberlemeye gerek yok, her iterasyonda parantez icindeki $1$ onceki faizi aliyor ve parantezin geri kalani ayni degere faiz uyguluyor. Bu parantezlerden carpim halinde $n$ tane oluyor.

\subsection{Probability}
\begin{example}
If $P(A) = 0.60$ then what is the highest value $P(B)$ can have if $A$ and $B$ are mutually exclusive?
\end{example}
\begin{explanation}
Highest value of $P(B) = 0.40$. Mutually exclusive means that if $A$ is happening $B$ wont happen, therefore $B$ can only happen $40\%$ of the time.
\end{explanation}
\textbf{Complement Rule}: $P(\text{not }A) = 1 - P(A)$ \par
If $P(A) = 1$ then we say event $A$ is completely certain: it is absolutely guaranteed that it will happen.

\par

FOr general events, $P(A \lor B) = P(A) + P(B) - P(A \land B)$


\subsection{Statistics} 
\textbf{Negatively correlated}: If two variables are negatively correlated, then in general, as one increases the other decrease. The graph would have a general negative-slope trend. \par

\textbf{Positively correlated}: If two variables are positively correlated, then in general, as one increases the other increases too. The graph would have a general positive-slope trend. \par

\textbf{Median}: Middle number on the ordered list. (If the list has even number of entries, the median is the average of the middle two numbers.) \par

\textbf{Mode}: The most frequent entry on a list, if each entry appears once then there is no mode. If multiple numbers are tied for most appearance, they are all modes. \par

\textbf{Mean}: Average of the entries in the list. \par 

\textbf{Range}: $max - min$ \par

\textbf{Fundamental Counting Principal}: If task 1 can happen in $n_1$ ways, task 2 in $n_2$ ways and so on for $m$ events, the number of outcomes is
$$
n_1\times n_2 \times ... \times n_m
$$

\textbf{nCr} ne demek? $C(n,r)$ ile ayni. ($n$ choose $r$). Number of combination of $r$ things that can be selected from a pool of $n$ things.

\textbf{Profit = Revenue - Cost}, yani kar esittir gelir eksi masraf.
\subsection{Geometry}
If a line has a negative y-intercept (where the line crosses y-axis), it must pass through which two quadrants?: III and IV. \par

If a line has a negative slope, it  must pass through which two quadrants? II and IV. \par 

Slopes of perpendicular lines are opposite reciprocals:
$$
m_1 = \frac{a}{b} \xrightarrow{\text{opposite reciprocal}} m_2 = -\frac{b}{a}
$$
Slopes of parallel lines are equal.
Slope of a vertical line is \textbf{undefined}.
Slope of a horizontal line is $\mathbf{0}$
\textbf{Regular Polygon}: Her kenar ayni uzunlukta ve her koseler ayni acida.
\textbf{Isosceles Triangle}: Ikizkenar Ucgen
\textbf{Equilateral Triangle}: Eskenar Ucgen
\textbf{Right-angled Triangle}: Dik-acili Ucgen
\textbf{Perimeter}: Cevre
\textbf{Rhombus}: Eskenar Dortgen (acilar 90 olunca kare oluyor, ama onun disinda herhangi bir sekilde olabilir)
\textbf{Diameter}: Cap
\textbf{Radius}: Yaricap
\textbf{Perimeter}: Cevre
\textbf{Circumference}: Cevre (Daire)
\textbf{N-kenar poligon ic acilar toplami}: $(n-2)\times 180^\circ$

\textbf{Guzel Pisagor ezberi}: $3,4,5$ var, $5,12,13$ var, $7,24,25$ var ve $8,15,17$.

An angle inscribed in a semicircle is always a $90^\circ$ angle. (Capi goren aci 90 derece)
\subsection{Numbers}

\textbf{Mixed Number}
$6\frac{3}{5}$ bir mixed numberdir. Ek olarak, $\frac{18}{5}$ ayni deger olsa da buna \textbf{Improper Fraction} denir. \par

\textbf{Proportion vs Ratio.} Ratio is a single fraction. Proportion is an equation with fractions on the both sides. \par

\textbf{Rounding} yaparken suna dikkat et: 5ten kucuk ise asagi, buyuk esit ise yukari yuvarlaniyor. \par

\textbf{Quotinent}: The result of a division is called quotinent. \par

When we divide $p$ by $q$, what is the name of the role q has, the role of the thing which we divide?: $q$ is \textbf{Divisor}, also $p$ is \textbf{Dividend}

\textbf{Finding Factors}: $N = p_1^{a_1}\times p_2^{a_2} \times ... p_n^{a_n}$ ise mesela ustlerinin bir fazlasinin carpimi ($p$ ler prime).
$$
#factors_of_N = (a_1+1)\times(a_2+1)\times...\times(a_n+1)
$$

Polinom hakkinda guzel bir sey:
$$
x^3-1=(x-1)(x^2+x+1)
$$

Roman numerals:
\begin{itemize}
    \item I = 1 (Unum)
    \item V = 5 (Quinque)
    \item X = 10 (Decem)
    \item L = 50 (Quinquaginta)
    \item C = 100 (Centum)
    \item D = 500 (Quingenti)
    \item M = 1000 (Mille)
\end{itemize}

Roman money currencies (denarius):
\begin{itemize}
    \item \sout{\textit{I}} = 1 (as)
    \item \sout{II} = 2 (Dupondious)
    \item \sout{IIS} = 2.5 (Sestertius)
    \item \sout{V} = 5 (Quinarius)
    \item \sout{X} = 10 (Denarius) (bu en cok kullanilan)
\end{itemize}

\end{document}
