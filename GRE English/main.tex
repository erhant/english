\documentclass[12pt, a4paper]{ximera}
\usepackage[utf8]{inputenc}
\usepackage[normalem]{ulem}
\usepackage{graphicx}
\usepackage{amsmath}
\usepackage{enumitem}

\title{Ingilizce Notlari by Erhan Tezcan}

\begin{document}

There is a competition \textbf{between} two people, but some competitions are \textbf{among} more than two competitors.

Fifteen items or \sout{less} \textbf{fewer}. (fewer is used for countables, less is used for uncountables)

English adjective order:
\begin{enumerate}
    \item Quantity or number
    \item Quality or opinion
    \item Size
    \item Age
    \item Shape
    \item Color
    \item Proper adjective (often nationality, other place of origin, or material)
    \item Purpose or qualifier
\end{enumerate}

\textbf{Duode}- is a prefix for twelve.

Had I known, I would have visited earlier (``had i known'' is a good alternative to ``if i had known'')

Were I to become President, I would make many changes. (``were I to become X'' is a good alternative to ``if I were to become X'')

The smartest student in the class is \sout{her} \textbf{she}.

After lunch, I will \sout{lay} \textbf{lie} down.

Yesterday, I lay down after lunch. (The verb "to lie" means "to recline. (geriye yaslanmak)" The past tense of "lie" is "lay." (Yes, the past tense of one verb is identical to the present tense of another verb: that is precisely why people confuse them!))

Drive \sout{like} \textbf{as if} your children \sout{live} \textbf{lived} here (this is wrong!, Correct versions include: "Drive as if your children lived here" or "Drive as you would if your children lived here." We use "like" to compare nouns, not actions \& situations.)

The picnic was canceled, \sout{due to} \textbf{because of} rain.

Hello, it is \sout{me} \textbf{I}.

A semi-colon is a conjunction that connects two independent (but related) clauses. ``She denied the accusation; Melissa had been at the movies at the time of the crime.''. On another example, ``A long-time proponent of nature conservation; the senator was excited to sponsor the wildlife refuge legislation.'' is wrong because the sentence before the semicolon does not contain a verb therefore is not an independent clause.

The Literature professor said that she considered Herman Melville \sout{to be} the single greatest American fiction writer.

Many Americans consider Labor Day \sout{as} the end of summer.

The verb to \textbf{solve} is generally used to mean find a solution – for example an answer or explanation. We solve a problem, something with a logical or complete answer. The conclusion reached with solving a problem, however, suggests the correct and definite answer has been found.

The verb to \textbf{resolve} has a number of meanings, one of which is to deal with conclusively – that is, to settle something, effectively to finish it in an acceptable way. This meaning of resolve is close to the meaning of solve, but with the difference that solve is used to find the correct answer to a problem; resolve is used more generally to conclude a problem. The conclusion reached with resolving something may be one of many choices, and it may not please everyone, but it concludes the problem, finishing it. 

Modern physical theories, as exemplified by quantum mechanics, contain unworldly ideas that sound fantastically unreal to the non-specialists. (``P, as exemplified by Q means'' that ``Q is an example of P'')

Forbid A to do X. (forbid A from doing X yanlis olur)

A is to B as X is to Y. (``a sock is to a foot as a glove is to a hand'' veya ``finding is to losing what building is to destroying'')

Not so much P as Q (Q'nun P oldugu kadar degil)

The young Joseph Campbell was fascinated \sout{by} \textbf{with} the Eskimo masks at the Museum of Natural History.

If yerine whether kullanirken bahsi gecen konu open-ended mi diye bakilir.

Statistics \sout{present} \textbf{presents} the only mathematical challenges in the otherwise relatively verbal field of psychology. (neden cogul kullanabildik? cunku Statistics cogul degil, singular ve bir Subject belirtiyor ders konusu gibi)



Either the legislators or the President \sout{are} \textbf{is} able to propose legislation, but only Congress can vote the bill into law. (In a "either A or B" construction, the verb must agree with the second term, the term closer to the verb. Since the singular noun "the President" is closer to the verb, we need the singular verb.)

The \sout{amount} \textbf{number} of people who had spilled into the publis square to witness the orator was staggering

Shirley would have \sout{ran} \textbf{run} in the marathon had she not twisted her ankle the night before the race.

An analysis of thousands of stars \sout{suggest} \textbf{suggests} that the number of exoplanets is greater than previously thought.

The unmarked envelope in the mailbox was neither for him nor \sout{I} \textbf{me}.

\sout{On account of him being the} \textbf{Because he was} late, the teacher did not allow him to take the test.

\textbf{Conjunction}: Linking two clauses.

\textbf{Gerund}: Verb form which functions as a noun (mastar eki almis fiil gibi, ingilizcede -ing ile biten mesela read ile reading gibi oluyor)

\textbf{Subordinate Clause}: A clause that modifies the principal clause (or some portion thereof) or that serves as a noun in a principal clause. ``I was not aware \textbf{that she was not planning on attending the party.}''

\textbf{Prepositional Phrase}: A phrase consisting of a preposition and an object, sometimes with other modifiers, that itself modifies another word or phrase. ``Stephen kept important mementos \textbf{in a tattered old box.}''

\textbf{Adverbial Nouns}: A noun used as an adverb; it often describes amount, time, distance, or some other measure. ``I will go to the store \textbf{tomorrow.}''

\textbf{Subject}: The part of the sentence that contains a noun (or noun substitute) and that indicates what in the sentence is doing what the sentence describes. ``\textbf{The loud, squawking birds} flew quickly overhead.''

\textbf{Predicate}: The part of the sentence that contains the verb (and any words that are governed by or modify the verb) and that expresses what the subject of the sentence is doing. ``The birds \textbf{flew quickly overhead, squawking as they went.}''

An adverb can modify an adjective too. (truly beatiful mesela...)

Experience abroad can be helpful for candidates \sout{,} \textbf{;} however that experience should be relevant to their careers they hope to pursue.

Gayle enjoys fishing, swimming and \sout{she likes to} bike. (Since the nouns are parallel writing likes to again would be extra)

Because her mother had been a lawyer, she had insight \sout{about} \textbf{into} the legal profession.

Boston, like Chicago, is famous for \sout{their} \textbf{its} passionate baseball fans.

The cheetah, the fastest mammal, is known for \sout{their} \textbf{its} tendency to tire quickly. 

Everyone should travel; it really \sout{opens} \textbf{broadens} the mind.

I like making things with my own hands. It gives me a lot of \sout{fulfillment} \textbf{satisfaction}.

We had to cover the rest of the \sout{contents} \textbf{syllabus} ourselves while our teacher was ill.

Air \sout{travelling} \textbf{travel} is cheaper than other forms of long-distance transport in my country.

The majority of cave art was created in prehistoric \sout{era} \textbf{times}.

It must have been amazing to be the first astronauts \sout{to} in space.

I don't think we will ever find another planet that can \sout{supply} sustain life.

My first job was to arrange the files into \sout{historical} chronological order from oldest to the most recent.

The tallest building where I live has 75 \sout{stories} storeys.

We really need to find a way to \sout{solve} resolve this issue.

The bad weather and a lack of food simply \sout{enhanced} compounded our problems.

Our new house was designed \sout{in} by computer.

Given the rapid growth of our population, there is a \sout{huge} pressing need to improve our infrastructure.

After a few hours of discussion we finally reached a \sout{solution} compromise.

People would use a lot less power if they bought household goods that were energy \sout{ecological} efficient.

I am writing to complain about a recent problem I had with a member of your \sout{workplace} staff.

I don't agree with \sout{enforcing} imprisoning very young or pretty criminals.

I think following celebrities on their holidays is an \sout{intrusion} invasion of privacy.

The ``LL'' rule when adding ``LY'' to the word:
\begin{itemize}
    \item If the word does not have ``L'' at the end we add ``LY'' and result in just one ``L''. (easy -> easily)
    \item If the word does have a ``L'' at the end we still add ``LY'' and results in ``LLY''. (real -> really)
    \item If the word has a ``LL'' at the end we add ``Y'' and result in ``LLY''. (full -> fully)
\end{itemize}

I hope that you will \sout{excuse} forgive me for behaving in this way.

The following year this figure \sout{drops} dropped noticeably from 72\% to only 55\%.

You need to choose a new assignment topic - this one is not \sout{unsimilar} dissimilar to the one you chose last term.

Table chart vs gosterirken \textbf{in} kullanilacak.

Some prepositions regarding dates:
\begin{itemize}
    \item \textbf{in + month or year} - In March, In 2003
    \item \textbf{on + date (with the year or without it) or day of the week} - On April 2, On March 3, 1999, On Saturday
    \item \textbf{at + clock time, midnight, noon} - At 3:30 p.m., At 4:01, At noon
    \item \textbf{in + season} - In the summer, In the winter
    \item \textbf{in + morning, afternoon, evening} - In the morning, In the evening
    \item \textbf{at + night} - At night
\end{itemize}

\textbf{Cosmopolitan}: familiar with and at ease in many different countries and cultures, global in reach and outlook. ``There are few cities in the world as diverse and cosmopolitan as New York.''

\textbf{Allotted}: Give or apportion (something) to someone as a share or task. ``Equal time was allotted to each''

\textbf{Staid}: characterized by dignity and propriety, sedate, respectable, and unadventurous. ``Frank came from a staid environment, so he was shocked that his college roommate sold narcotics.''

\textbf{Solemn}: Formal and dignified (agirbasli). `` Finally he sat up to face her, his eyes now solemn and serious.''

\textbf{Lewd}: Crude and offensive in a sexual way. ``She began to gyrate to the music and sing a lewd song.''

\textbf{Expurgate}: remove matter thought to be objectionable or unsuitable from (a book or account). ``The censor expurgated every reference to sex and drugs, converting the rapper's raunchy flow into a series of bleeps.''

\textbf{Banal}: So lacking in originality as to be obvious and boring. ``Songs with banal, repeated words.''

\textbf{Segue}: (in music and film) move without interruption from one piece of music or scene to another. ``Going from high-school to college like a segue, which results in students unaware of the seriosity of the lectures''.

\textbf{Leeway}: the amount of freedom to move or act that is available. ``The government had several months' leeway to introduce reforms''

\textbf{Upbraid}: scold harshly. ``He had to upbraid the apparent drunk barber.''

\textbf{Denigrate}: criticize unfairly; disparage, dispraise. ``there is a tendency to denigrate the poor.''

\textbf{Lambaste}. Criticize harshly. ``Showing no patience, the manager utterly lambasted the sales team that lost the big account.''

\textbf{Credence}: belief in or acceptance of something as true. ``psychoanalysis finds little credence among laymen.''

\textbf{Credulity}: a tendency to be too ready to believe that something is real or true. ``At one extreme, it is used to represent the unswerving conviction and absolute credulity of the true believer.''

\textbf{Tokenism} The practice of making only a perfunctory or symbolic effort to do a particular thing, especially by recruiting a small number of people form underrepresented groups in order to give the appearance of sexual or racial equality within a workforce. ``Simply electing a few women to assume key political positions is mere tokenism, and should fool no one.''

\textbf{Perfunctory}: (of an action or gesture) carried out with a minimum of effort or reflection. ``The short film examines modern perfunctory cleaning rituals such as washing dishes, doing laundry and tooth-brushing.''

\textbf{Blunt}: Make or become less sharp; (of a person or remark) uncompromisingly forthright candid / crude / straightforward. ``During short conversation with the reporter, he was somewhat blunt but at last he was quite gentle in tone.

\textbf{Forthright}: (of a person or their manner or speech) direct and outspoken; straightforward and honest. ``Despite his forthright views, however, he remains unfailingly courteous to the end.''

\textbf{Unforthcoming}: uncooperative, not willing to give up information. ``The teacher demanded to know who broke the window while he was out of the room, but the students understandably were unforthcoming.''

\textbf{Forthcoming}: (of something required) ready or made available when wanted or needed. ``financial support was not forthcoming.''

\textbf{Crude}: In a natural or raw state; not yet processed or refined. ``Crude oil''

\textbf{Rarefied}: made more subtle or refined. ``Jack's vulgar jokes were not so successful in the rarefied environment of college professors.''

\textbf{Leisure}: Free time. ``Usually the leisure consists of snorkeling, night club jaunts and ample free time spent with other American youths.''

\textbf{Recreational}: relating to or denoting activity done for enjoyment when one is not working. ``You are asked to list recreational interests and activities, membership of clubs and societies.''

\textbf{Remonstrate}: make a forcefully reproachful protest. ``he turned angrily to remonstrate with Tommy.''

\textbf{Invective}: insulting, abusive, or highly critical language. ``he let out a stream of invective.''

\textbf{Censure}: the expression of formal disapproval. (kinama). ``angry delegates offered a resolution of censure against the offenders.''

\textbf{Chastise}: to rebuke or reprimand severely. ``he chastised his colleagues for their laziness.''

\textbf{Lampoon}: ridicule with satire, a speech or text criticizing someone or something in this way. ``Mark Twain understood that lampooning a bad idea with humor was the most effective criticism.''

\textbf{Castigate}: to reprimand harshly. ``Drill sergeants are known to castigate new recruits so mercilessly that the latter often break down during their first week in training.''

\textbf{Excoriate}: censure or criticize severely. ``the papers that had been excoriating him were now lauding him.''

\textbf{Rebuke}: an expression of sharp disapproval or criticism. ``he hadn't meant it as a rebuke, but Neil flinched.''

\textbf{Objurgate}: express strong disapproval of. ``The manager spent an hour objurgating the employee in the hopes that he would not make these mistakes again.''

\textbf{Execrate}: feel or express great loathing for, curse, hiss at. ``Didn't Trotsky execrate those who claimed to believe there was nothing to choose between democracy and fascism?.''

\textbf{Loath}: reluctant; unwilling. ``I was loath to leave.''

\textbf{Vituperate}: blame or insult (someone) in strong or violent language. ``In most cases, a cold dose of healthy public ridicule would quench (take our fire) the more volcanic vituperators and reason would be restored.''

\textbf{Reprimand}: a rebuke, especially an official one. ``The destruction of the Afghan Buddhas was met with reprimands from our officials, while ancient religious sites in our own country are being turned into quarries.''

\textbf{Abrogate}: revoke or relinquish formally; do away with. ``As part of the agreement between the labor union and the company, the workers abrogated their right to strike for four years in exchange for better health insurance.''

\textbf{Countermand}: a contrary command cancelling or reversing a previous command. ``By the time the colonel countermanded his soldiers not to land in enemy territory, a few helicopters had already touched down amid heavy gunfire.''

\textbf{Kowtow}: act in an excessively subservient manner. (secde). ``Paul kowtowed to his boss so often the boss herself became nauseated by his sycophancy.''

\textbf{Approbatory}: expressing praise or approval. ``Although it might not be her best work, Hunter's new novel has received generally approbatory reviews.''

\textbf{Fete}: To celebrate a person. ``After WWII, war heroes were feted first but quickly forgotten.''

\textbf{Lauding}: praise (a person or their achievements) highly, especially in a public context. ``the obituary lauded him as a great statesman and soldier.''

\textbf{Vaunt}: boast about or praise (something), especially excessively. ``the much vaunted information superhighway.''

\textbf{Fawn}: (of a person) give a servile display of exaggerated flattery or affection, typically in order to gain favor or advantage. ``congressmen fawn over the President.''

\textbf{Hagiographic}: excessively flattering toward someone's life or work. ``Most accounts of Tiger Woods's life were hagiographic, until, that is, his affairs made headlines.''

\textbf{Accrete}: grow by accumulation or coalescence. ``ice that had accreted grotesquely into stalactites.''

\textbf{Apotheosis}: the highest point in the development of something; culmination or climax. ``his appearance as Hamlet was the apotheosis of his career.''

\textbf{Caveat}: a warning or proviso of specific stipulations (rules), conditions, or limitations. (yani ek bir uyari gibisinden)
``But it is also liberally sprinkled with caveats and warnings as to the difficulties in turning up more evidence.''

\textbf{Tarnish}: make dirty or spotty, as by exposure to air; also used metaphorically. ``Pete Rose was one of the best baseball players of his generation, but his involvement with gambling on baseball games has tarnished his image in the eyes of many.''

\textbf{Patronize}: treat with an apparent kindness that betrays a feeling of superiority; frequent (a store, theater, restaurant, or other establishment) as a customer. ``And, of course, avoid anyone who is patronizing or condescending.'' - ``restaurants remaining open in the evening were well patronized.'' 

\textbf{Peremptory}: (especially of a person's manner or actions) insisting on immediate attention or obedience, especially in a bossy, brusquely imperious way. ``“Just do it!” came the peremptory reply.''

\textbf{Imperious}: assuming power or authority without justification; arrogant and domineering. ``his imperious demands.''

\textbf{Despot}: a cruel and oppressive dictator. ``The Emperor Claudius was regarded as a fair-minded leader; his successor, Nero, was an absolute despot.''

\textbf{Browbeat}: intimidate (someone), typically into doing something, with stern or abusive words. ``a witness is being browbeaten under cross-examination.''

\textbf{Broadside}: a strong verbal attack / side by side fight of cannon-ships. ``Political broadsides are usually strongest in the weeks leading up to a national election.''

\textbf{Strife}: angry or bitter disagreement over fundamental issues; conflict. ``strife within the community.''

\textbf{Diatribe}: a forceful and bitter verbal attack against someone or something. ``a diatribe against the Roman Catholic Church.''

\textbf{Elegy}: A poem of serious reflections, typically a lament (a\u{g}{\i}t) for the dead. ``Addison was buried in Westminster Abbey, and lamented in an elegy by Tickell.''

\textbf{Pine}: To yearn for. ``She pined for her lost love.''

\textbf{Quaff}: Drink something heartily, usually alcohol. ``If you mix three different rums with four kinds of fruit juice, the chances that the finished product will be a pleasant quaff are pretty good.''

\textbf{Imbibe}: drink alcohol. ``The company claims that if you take this pill, you will need less alcohol to stay drunk, so will imbibe less.''

\textbf{Veneer}: Cover (something) with a layer. ``Her veneer of composure cracked a little.''

\textbf{Ineluctable}: unable to be resisted or avoided; inescapable. ``the ineluctable facts of history.''

\textbf{Desist}: to stop doing; cease; abstain. ``each pledged to desist from acts of sabotage.''

\textbf{Inexorable}: Impossible to stop or prevent. ``Death is inexorable.''

\textbf{Obdurate}: stubbornly refusing to change one's opinion or course of action. ``But I think it saddened him to see people obdurate , unwilling to let go of doctrinaire positions instead of facing issues on their merits.''

\textbf{Reprieve}: A cancellation / postponement of a punishment; relieve. ``under the new regime, prisoners under sentence of death were reprieved.''

\textbf{Renege}: Go back on a promise / undertaking / contract;  fail to fulfill a promise or obligation. ``The government had reneged on its election promises, and its not like Turkish voters care about those promises anyway.''

\textbf{Malfeasance}: misconduct or wrongdoing (especially by a public official). ``This would occur whenever the public is made aware of official malfeasance or incompetence.''

\textbf{Misdemeanour}: A minor wrongdoing. ``The player can expect a suspension for his latest misdemeanour.'' - ``Pardon me, for my misdemeanour. I am stranger to this activity.''

\textbf{Perfidy}: deceitfulness; untrustworthiness. ``I have been accused of perfidy , malingering, duplicity, charlatanism and forty other words that I don't know the meaning of.''

\textbf{Antic}: Grotesque or bizarre, comically or repulsively ugly or distorted. (Antic daha komik olan hali icin). ``An antic disposition.''

\textbf{Meander}: To wander aimlessly. ``A couple of tracks meander aimlessly so that by the end of the album I'm left with the feeling that the band didn't quite know which direction to take.''

\textbf{Maunder}: wander aimlessly; speak (about unimportant matters) rapidly and incessantly. ``Dennis maundered on about the wine.''

\textbf{Flounder}: behave awkwardly; have difficulties. ``Sylvia has excelled at advanced calculus, but ironically, when she has to deal with taxes, she flounders.''

\textbf{Disposition}: A person's inherent qualities of mind and character / the way in which something is placed or arranged, especially in relation to other things. ``A sweet-natured girl of a placid disposition.''

\textbf{Placid}: (of a person or animal) not easily upset or excited.

\textbf{Reprisal}: An act of retaliation (misilleme, aynen karsilik verme). ``Three youths died in the reprisals that followed''.

\textbf{Recrimination}: an accusation in response to one from someone else. ``Matters aren't helped by the fact that the bitter recriminations continue to rumble on.''

\textbf{Sporadic}: Recurring in scattered and irregular or unpredictable instances. ``The signals were at first sporadic, but now we detect a clear, consistent pattern of electromagnetic radiation emanating from deep space.''

\textbf{Interloper}: A person who becomes involved in a place or situation where they are not wanted or are considered not to belong. ``I am a stranger, an interloper, who does belong in this amazing, fantastical world.''

\textbf{Ominous}: giving a feeling that something bad will happen, inauspicious. ``there were ominous dark clouds gathering overhead.''

\textbf{Apprehension}: Anxiety or fear that something bad or unpleasant will happen. ``He felt sick with apprehension.'' - ``Every being (living thing) will taste death, at the end of a life-time which is ephemeral relative to the time of existence all together. Should this be a source of apprehension, or should it be a comfort that everything is leading to the same conclusion.''

\textbf{Perennial}: lasting or existing for a long or apparently infinite time; enduring or continually recurring. ``his perennial distrust of the media.''

\textbf{Transitory}: Lasting a very short time. ``If we lived forever and life was not transitory, do you think we would appreciate life less or more?''

\textbf{Ephemeral}: lasting for a very short time. ``fashions are ephemeral.''

\textbf{Respite}: a short period of rest or relief from something difficult or unpleasant. ``the refugee encampments will provide some respite from the suffering.''

\textbf{Leery}: cautious or wary due to realistic suspicions. ``a city leery of gang violence.''

\textbf{Fauna}: All the animals that live in a particular area.

\textbf{Flora}: All the plants that live in a particular area.

\textbf{Discursive}: Digressing from subject to subject, tangential. ``Students often write dull, secondhand, discursive prose.''

\textbf{Digress}: Leave the main subject temporarily in speech or writing. ``I often digress in my writings, I am aware of this problem yet it keeps happening!''

\textbf{Snide}: Derogatory or mocking in an indirect way (kucumseyici). / Counterfeit; inferior. ``He was snide too, bringing up how much of a hound Mitch was and how that might unhinge any woman's mind.''

\textbf{Derogatory}: showing a critical or disrespectful attitude. ``she tells me I'm fat and is always making derogatory remarks.''

\textbf{Supercilious}: behaving or looking as though one thinks one is superior to others. ``Darcy, though attracted to the next sister, the lively and spirited Elizabeth, greatly offends her by his supercilious behaviour at a ball.''

\textbf{Goad}: provoke or annoy (someone) so as to stimulate some action or reaction. ``he goaded her on to more daring revelations.''

\textbf{Tact}: consideration in dealing with others and avoiding giving offense. ``the inspector broke the news to me with tact and consideration.''

\textbf{Tactful}: having or showing tact. ``they need a tactful word of advice.''

\textbf{Solecism}: a socially awkward or tactless act. ``'' - ``It is regarded as a solecism to say ‘We have less tea bags than I thought.’''

\textbf{Gaffe}: an unintentional act or remark causing embarrassment to its originator; a blunder. (gaf). ``an unforgivable social gaffe.''

\textbf{Unhinge}: Make (someone) mentally unbalanced. ``The loneliness had nearly unhinged him.''

\textbf{Sartorial}: of or relating to tailoring, clothes, or style of dress. ``sartorial elegance.''

\textbf{Genteel}: Polite, refined, respectable, often in an affected or ostentatious way. ``After that it looks like something from a more refined and genteel and luxurious and over-the-top era.''

\textbf{Ostentatious}: Characterized by vulgar or pretentious display; designed to impress or attract notice (gosterisli). ``Books that people buy and display ostentatiously but never actually finish...''

\textbf{Quaint}: attractively unusual or old-fashioned, antique. ``quaint country cottages.''

\textbf{Vulgar}: lacking sophistication or good taste; unrefined. ``The most common forms of abuse were much less sophisticated and amounted to little more than vulgar name-calling.''

\textbf{Tout}: attempt to sell (something), typically by pestering people in an aggressive or bold manner; advertise in strongly positive terms; show off. ``Khajit was touting his wares.''

\textbf{Harrowing}: Acutely distressing. ``A harrowing film about racism and violence.''

\textbf{Appalling}: Awful, terrible (dehset verici). ``His conduct was appalling.''

\textbf{Egregious}: outstandingly bad; shocking / remarkably good. ``egregious abuses of copyright.'' - ``I am not so egregious a mathematician as you are.''

\textbf{Conduct}: The manner in which a person behaves, especially on a particular occasion or in a particular context.

\textbf{Aghast}: Filled with horror or shock. ``The church volunteers who serve it were aghast and flabbergasted.''

\textbf{Flabbergasted}: (hayrete dusmus) ``His conduct left me flabbergasted.''

\textbf{Surfeit}: an excessive amount of something / cause (someone) to desire no more of something as a result of having consumed or done it to excess. ``a surfeit of food and drink.'' - ``I am surfeited with shopping.''

\textbf{Superfluous}: Unnecessary, especially through being more than enough, extravagant. ``The purchaser should avoid asking for superfluous information.''

\textbf{Overweening}: Showing excessive confidence or pride, arrogant, presumptuous. ``Overweening ambition.''

\textbf{Hubris}: excessive pride or self-confidence. ``Arrogance, hubris , blind patriotism, and good old fashioned fear are our real enemy!''

\textbf{Taciturn}: (of a person) reserved or uncommunicative in speech; saying little. ``Being taciturn in a speech exam is obviously not the best conduct.''

\textbf{Rapprochement}: (especially in international relations) an establishment or resumption of harmonious relations. (uzlasma). ``There were signs of growing rapprochement between the two countries.''

\textbf{Reproach}: the expression of disapproval or disappointment. (sitem). ``At first, Sarah was going to yell at the boy, but she didn't want to reproach him for telling the truth about the situation.''

\textbf{Hedge}: A fence or boundary / surround or bound with a hedge / limit or qualify (something) by conditions or exceptions. ``Experts usually hedge their predictions, just in case.''

\textbf{Snub}: An act of showing disdain or a lack of cordiality by rebuffing or ignoring someone or something. ``He couldn't help thinking that the whole thing was meant to be taken as a snub.''

\textbf{Cordiality}: Amity (a friendly relationship) (samimiyet). ``You must be deliberately ignorant to not notice the disingenuous cordiality among the group. Now the question is not whether if there will be an outbreak, it is when?''

\textbf{Cordially}: (candan, cana yakin sekilde). ``You are cordially invited.''

\textbf{Disdain}: The feeling that someone or something is unworthy of one's consideration or respect; contempt. ``her upper lip curled in disdain.''

\textbf{Boorish}: rough and bad-mannered; coarse. ``boorish behavior.''

\textbf{Coarse}: (of a person or their speech) rude, crude, or vulgar. ``You are never coarse or vulgar, and people who display such traits offend you.''

\textbf{Churlish}: rude in a mean-spirited and surly way. ``it seems churlish to complain.''

\textbf{Baseness}: lack of moral principles; bad character. ``the baseness of human nature.''

\textbf{Sordid}: involving ignoble actions and motives; arousing moral distaste and contempt. ``the story paints a sordid picture of bribes and scams.''

\textbf{Sardonic}: grimly mocking or cynical, disdainfully or ironically humorous; scornful and mocking. ``A stand-up comedian walks a fine line when making jokes about members of the audience; such fun and joking can quickly become sardonic and cutting.''

\textbf{Contempt}: The feeling that a person or a thing is beneath consideration, worthless, or deserving scorn. ``he showed his contempt for his job by doing it very badly.''

\textbf{Pillory}: ridicule or expose to public scorn. ``After the candidate confessed, the press of the opposing party took the opportunity to pillory him, printing editorials with the most blatantly exaggerated accusations.''

\textbf{Deride}: express contempt for; ridicule. ``critics derided the proposals as clumsy attempts to find a solution.''

\textbf{Spurn}: Reject with disdain or contempt. ``He spoke gruffly, as if afraid that his invitation would be spurned.''

\textbf{Carping}: difficult to please; critical; complain or find fault continually, typically about trivial matters. ``she has silenced the carping critics with a successful debut tour.'' - ``I don't want to carp about the way you did it.''

\textbf{Gruff}: Abrupt (sudden, taking short time) or taciturn in manner. ``Penetrate a gruff exterior and you will find him affable.'' - ``Most geeks manifest a gruff exterior, however, converse on their topic and you will find yourself bombarded with information regarding the topic.''

\textbf{Immure}: Enclose or confine someone against their will. ``Her brother was immured in a lunatic asylum.'' - ``I don't want to sound racist but I cant help but feel immured by the Syrians coming to my hometown. I'm sure hearing their language instead of mine on a regular basis is a major cause of this feeling.''

\textbf{Inure}: accustom (someone) to something, especially something unpleasant. ``these children have been inured to violence.''

\textbf{Dispensation}: exemption from a rule or usual requirement. ``although she was too young, she was given special dispensation to play two matches.''

\textbf{Exemption}: The process of freeing or state of being free from an obligation or liability imposed on others (muafiyet). ``I am exempted from taking English courses thanks to my performance on the English proficiency test.''

\textbf{Callow}: (especially of a young person) inexperienced and immature. ``earnest and callow undergraduates.''

\textbf{Tyro}: someone new to a field or activity, a beginner or novice. ``All great writers, athletes, and artists were tyros at one time—unknown, clumsy, and unskilled with much to learn.''

\textbf{Dilettante}: A person who cultivates an area of interest, such as the arts, without real commitment or knowledge. (Amator). ``He realized that being dilettante on many areas does not necessarily make him multi-disciplinary.''

\textbf{Protean}: tending or able to change frequently or easily. ``it is difficult to comprehend the whole of this protean subject.''

\textbf{Cultivate}: Try to acquire and develop (a quality, sentiment, or skill). ``It has given him time to cultivate his skills and experiment in a more relaxed environment than at home, where everyone's expectations are higher than average.''

\textbf{Asperity}: harshness of tone or manner. ``he pointed this out with some asperity.''

\textbf{Austere}: Severe or strict in manner, attitude or appearance. / practicing self-denial. / unadorned (not adorned; plain.) ``Peter expected high standards, but his sometimes austere manner veiled a deep concern for people and an insight into the human condition.'' - ``His lifestyle of revelry and luxurious excess could hardly be called austere.''

\textbf{Brazen}: bold and without shame. ``he went about his illegal business with a brazen assurance.''

\textbf{(to) Veil}: Cover with or as though with a veil. ``shrouded in an eerie veil of mist.''

\textbf{Trenchant}: vigorous or incisive in expression or style. ``As social critics, they are trenchant and savage, just as one might expect of two former art students who cut their teeth (gain experience) on the Sex Pistols and the Situationists.''

\textbf{Incisive}: (of a person or mental process) intelligently analytical and clear-thinking. ``She was an incisive critic.''

\textbf{Incandescent}: passionate or brilliant; very hot. (akkor). ``Mravinsky's incandescent performance of Siegfried's Funeral March.''

\textbf{Leniency}: The fact or quality of being more merciful or tolerant than expected; clemency (hosgoru). ``The court could have shown leniency.''

\textbf{Inclement}: used of persons or behavior; showing no mercy. ``Marcus Aurelius, though a fair man, was inclement to Christians during his reign, persecuting them violently.''

\textbf{Tempestuous}: characterized by strong and turbulent or conflicting emotion. ``he had a reckless and tempestuous streak.''

\textbf{Demented}: behaving irrationally due to anger, distress or excitement. ``She was demented with worry.''

\textbf{Umbrage}: a feeling of anger caused by being offended. ``Since he was so in love with her, he took umbrage at her comments, even though she had only meant to gently tease him.''

\textbf{Cosset}: care for and protect in an overindulgent way. ``all her life she'd been cosseted by her family.''

\textbf{Indulge}: Allow oneself to enjoy the pleasure of. ``I indulged in fast food, voraciously.

\textbf{Ravenous}: extremely hungry; devouring or craving food in great quantities. ``A ravenous apetite.''

\textbf{Resurgent}: rising again as to new life and vigor. ``The team sank to fourth place in June, but is now resurgent and about to win the division.''

\textbf{Peril}: Serious and immediate danger. ``The perils of division by 0.'' - ``You could well place us both in peril.''

\textbf{Prosecute}: institute legal proceedings against (a person or organization) / continue with (a course of action) with a view to its completion. 

\textbf{Confound}: Cause surprise or confusion in (someone) especially by acting against their expectations / mix up (something) with something else so that the individual elements become difficult to distinguish. ``The inflation figure confounded economic analysts'' - ``Do not confound it with cowardice or ill-temper.''

\textbf{Soporific}: tending to induce (cause to happen) drowsiness or sleep. ``the motion of the train had a somewhat soporific effect.''

\textbf{Repudiate}: Refuse to accept or be associated with, deny as untrue. ``She has repudiated policies associated with previous party leaders.''

\textbf{Patent}: Easily recognizable; obvious. ``She was smiling with patent insincerity.''

\textbf{Pervasive}: (especially of an unwelcome influence or physical effect) spreading widely throughout an area or group of people. ``Ageism (prejudice or discrimination on the basis of a person's age) is pervasive and entrenched in our society.''

\textbf{Overlook}: Fail to notice something / have a view of from above. ``He seems to have overlooked one important fact.'' - ``The chateau overlooks fields of corn and olive trees.''

\textbf{Construe}: Interpret (a word or action) in a particular way. ``His words could hardly be construed as an apology.''

\textbf{Conflate}: combine (two or more texts, ideas, etc.) into one. ``In her recent book, the author conflates several genres--the detective story, the teen thriller, and the vampire romance--to create a memorable read.''

\textbf{Anoint}: Smear or rub with oil, typically as part of a religious ceremony. ``High priests were anointed with oil.''

\textbf{Idiosyncrasy}: a behavioral attribute that is distinctive and peculiar to an individual. ``Peggy's numerous idiosyncrasies include wearing mismatched shoes, laughing loudly to herself, and owning a pet aardvark.''

\textbf{Foible}: a minor weakness or eccentricity in someone's character. ``they have to tolerate each others little foibles.''

\textbf{Apocryphal}: (of a story or statement) of doubtful authenticity, although widely circulated as being true. ``an apocryphal story about a former president.''

\textbf{Undermine}: Damage or weaken (someone or something), especially gradually or insidiously. ``This could undermine years of hard work.''

\textbf{Collude}: Come to secret understanding for a harmful purpose, conspire. ``University leaders colluded in price-rigging.''

\textbf{Retention}: The continued possession, use, or control of something. ``The retention of direct control by central government.''

\textbf{Homophone}: Okunusu ayni anlami farkli (Mete ile meet mesela)

\textbf{Alliteration}: The occurrence of the same letter or sound at the beginning of adjacent or closely connected words.

\textbf{Forgery}: The action of forging or producing a copy of a document, signature, banknote, or work of art. (sahtecilik). ``He was convicted of forgery, attempted theft and perjury and will be sentenced on September 15.''

\textbf{Vitiate}: Spoil or impair the quality or efficiency of. ``Development programs have been vitiated by the rise in population.''

\textbf{Vacillate}: Alternate or waver between different opinions or actions; be indecisive. ``I had for a time vacillated between teaching and journalism.''

\textbf{Curmudgeon}: A bad-tempered or surly person. ``Only the worst curmudgeon could dislike this site.''

\textbf{Martinet}: A strict disciplinarian, especially in the armed forces. ``The woman in charge was a martinet who treated all those beneath her like children.''

\textbf{Ribald}: Referring to sexual matters in an amusingly rude or irreverent way. ``A ribald comment.''

\textbf{Jocular}: Characterized by jokes and good humor. ``she sounded in a jocular mood.''

\textbf{Facetious}: treating serious issues with deliberately inappropriate humor; flippant. ``Facetious behavior will not be tolerated during sex education class; it's time for all of you to treat these matters like mature adults.''

\textbf{Facet}: one side of something many-sided, especially of a cut gem or abstract things. ``After two years of living abroad, Mario decided that he missed some facets of American food culture, including the variety of restaurant types.''

\textbf{Quip}: a witty saying or remark. ``In one of the most famous quips about classical music, Mark Twain said: "Wagner's music is better than it sounds.''

\textbf{Facile}: arrived at without due care or effort; lacking depth. ``Many news shows provide facile explanations to complex politics, so I prefer to read the in-depth reporting of The New York Times.''

\textbf{Reverent}: feeling or showing profound respect or veneration. ``a reverent silence.''

\textbf{Forage}: (of a person or animal) search widely for food or provisions. ``gulls are equipped by nature to forage for food.''

\textbf{Provision}: the action of providing or supplying something for use. ``new contracts for the provision of services.''

\textbf{Provisional}: arranged or existing for the present, possibly to be changed later. ``a provisional government.''

\textbf{Tentative}: not certain or fixed; provisional. ``a tentative conclusion.''

\textbf{Wanton}: (of a cruel or violent action) deliberate and unprovoked. ``Sheer wanton vandalism''.

\textbf{Promiscuous}: Having or characterized by many transient sexual relationships. ``She is a wild, promiscuous girl.''

\textbf{Dissolution}: a living full of debauchery (excessive indulgence in sex, alcohol, or drugs.) and indulgence in sensual pleasure. ``Many Roman emperors were known for their dissolution, indulging in unspeakable desires of the flesh.''

\textbf{Lascivious}: (of a person, manner, or gesture) feeling or revealing an overt and often offensive sexual desire, lecherous. ``she ignored his lecherous gaze.''

\textbf{Bawdy}: Dealing with sexual matters in a comical way; humorously indecent.

\textbf{Irreverent}: Showing lack of respect for people or things that are generally taken seriously. ``She is irreverent about the whole business of politics.''

\textbf{Bellicose}: Demonstrating aggression and willingness to fight. ``A group of bellicose patriots.''

\textbf{Philistine}: A person who is hostile or indifferent to culture and the arts, or who has no understanding of them. ``I am a complete philistine when it comes to paintings.''

\textbf{Fecund}: Producing or capable of producing an abundance of offspring or new growth; fertile. ``A lush and fecund garden.''

\textbf{Profuse}: (especially of something offered or discharged) exuberantly plentiful; abundant. ``I offered my profuse apologies.''

\textbf{Lush}: (of vegetation) growing luxuriantly / heavy drinker / make someone drunk. ``Lush greenery and cultivated fields.'' 

\textbf{Moot}: subject to debate, dispute, or uncertainty, and typically not admitting of a final decision. ``whether the temperature rise was mainly due to the greenhouse effect was a moot point.''

\textbf{Bromide}: A trite and unoriginal idea or remark, typically intended to soother or placate. ``Feel good bromides create the illusion of problem solving.''

\textbf{Venality}:  is the quality of being open to bribery or overly motivated by money. (rusvet almaya aciklik). ``Venality of cops in Turkey.''

\textbf{Bastardization}: an act that debases or corrupts. ``The movie World War Z is a complete bastardization of the book with little more in common than zombies and a title.''

\textbf{Debase}: Reduce the quality or value of something. ``The third-rate script so debased the film that not even the flawless acting could save it from being a flop.''

\textbf{Denouement}: The final part of a play, or narrative in which the strands of the plot are drawn together and matters are explained or resolved. ``As the music built to a final denouement a bright city rose behind the dancers and they joyfully went to enter it.''

\textbf{Enamored}: be filled with a feeling of love for. ``it is not difficult to see why Edward is enamored of her.''

\textbf{Besotted}: Strongly infatuated, affectionate. / drunk;  ``This was the speech of a prime minister besotted with emotion, image and presentation.''

\textbf{Anodyne}: Not likely to provoke dissent or offense, inoffensive, often deliberately so. / a painkilling drug or medicine. ``Anodyne New Age music.'' / ``Overall, it is a little too anodyne and pleasant.''

\textbf{Factious}: produced by, or characterized by internal dissension. ``The controversial bill proved factious, as dissension even within parties resulted.''

\textbf{Dissent}: the expression or holding of opinions at variance with those previously, commonly, or officially held. (muhalefet). ``there was no dissent from this view.''

\textbf{Dissension}: disagreement that leads to discord. ``this maneuver caused dissension within feminist ranks.''

\textbf{Infatuated}: Be inspired with an intense but short-lived passion or admiration for. ``She is infatuated with a handsome police chief.''
aghast

\textbf{Heresy}: Belief or opinion contrary to orthodox religious (especially Christian) doctrine. ``Huss was burned for heresy''. 

\textbf{Blasphemy}: The act or offense of speaking sacrilegiously about God or sacred things; profane talk. ``He was detained on charges of blasphemy.''

\textbf{Sybarite}: A person who is self-indulgent in their fondness for sensuous luxury. ``They encouraged him, while in the city, to live like a sybarite.

\textbf{Guffaw}: a loud and boisterous laugh. ``Whenever the jester fell to the ground in mock pain, the king guffawed, exposing his yellow, fang-like teeth.''

\textbf{Apostate}: a person who renounces a religious or political belief or principle. ``We may earnestly believe that they're wrong - whether they're non-Christians, heretics, apostates , agnostics, atheists, or what have you.''

\textbf{Anathema}: Something or someone that one vehemently dislikes. / A formal curse by pope or a council of the Church, excommunicating a person or denouncing a doctrine. ``Racial hatred was anathema to her.''

\textbf{Ostracize}: exclude from a community or group. ``Later in his life, Leo Tolstoy was ostracized from the Russian Orthodox Church for his writings that contradicted church doctrine.''

\textbf{Pariah}: an outcast. ``The once eminent scientist, upon being found guilty of faking his data, has become a pariah in the research community.''

\textbf{Malediction}: (beddua). ``He muttered a quiet malediction.''

\textbf{Vicissitude}: change in one’s circumstances, usually for the worse. ``Even great rulers have their vicissitudes—massive kingdoms have diminished overnight, and once beloved kings have faced the scorn of angry masses.''

\textbf{Provincial}: of or concerning a province a country or empire. / (eski kafali, koylu) ``Provincial elections.'' - ``Scenes of violence were reported in provincial towns.''

\textbf{Industrious}: diligent and hard-working. ``He struck me as being particularly hardworking, energetic and industrious.''

\textbf{Artless}: without guile or deception. ``Despite the president's seemingly artless speeches, he was a skilled and ruthless negotiator.''

\textbf{Guileless}: Devoid of guile; innocent and without deception. ``His face, once so open and guileless.''

\textbf{Chagrin}: feeling of / to feel distressed or humiliated. ``Jeff, much to his chagrin , wasn't invited''

\textbf{Derisive}: Abusing vocally; expressing contempt or ridicule. ``I was surprised by her derisive tone; usually, she is sweet, soft spoken, and congenial.''

\textbf{Congenial}: (of a person) pleasant because of a personality, qualities, or interests that are similar to one's own. ``his need for some congenial company.''

\textbf{Genial}: agreeable, conducive to comfort. ``Betty is a genial young woman: everyone she meets is put at ease by her elegance and grace.''

\textbf{Cherish}: Protect and care for someone lovingly. ``He cared for me beyond measure and cherished me in his heart.''

\textbf{Inadvertent}: not resulting from or achieved through deliberate planning. ``an inadvertent administrative error occurred that resulted in an over-payment.''

\textbf{Fortuitous}: Happening by accident or chance rather than design. ``Much of the success of the text is by design, other aspects are by fortuitous accident.''

\textbf{Auspicious}: Conducive to success, favorable. ``It was not the most auspicious moment to hold an election.''

\textbf{Conducive}: making a situation or outcome more likely to happen. ``Studying in a quiet room is conducive to learning; studying in a noisy environment makes learning more difficult.''

\textbf{Infelicitous}: Inappropriate. ``During the executive meeting, the marketing director continued to make infelicitous comments about the CEO's gambling habit.''

\textbf{Untoward}: unexpected and inappropriate or inconvenient. ``both tried to behave as if nothing untoward had happened.''

\textbf{Turgid}: Swollen and distended or congested / (of language) pompous and tedious. ``A turgid and fast-moving river.'' - ``The amount of GRE vocabulary he used increased with his years--by the time he was 60, his novels were so turgid that even his diehard fans refused to read them.''

\textbf{Glib}: (of words or the person speaking them) fluent and voluble but insincere and shallow. ``She was careful not to let the answer sound too glib.''

\textbf{Piquant}: Having a pleasantly sharp taste or appetizing flavor. ``Herbs and spices add a piquant taste that ketchup can't match.''

\textbf{Recrudesce}: Break out again, recur. ``She was able to see a festering protest recrudescing.''

\textbf{Wary}: Feeling or showing caution about possible dangers or problems. ``Dogs that have been mistreated often remain very wary of strangers.''

\textbf{Raillery}: Good-humored teasing. ``She was greeted with raillery from her fellow workers.'' - ``A veneer of raillery does not evanesce the cherishing atmosphere of the family.''

\textbf{Evanescent}: soon passing out of sight, memory, or existence; quickly fading or disappearing. ``However, this skeptical triumph is evanescent, it vanishes when his attention turns to other facts.''

\textbf{Prose}: Written or spoken language in its ordinary form, without metrical structure (duz yazi) / talk tediously. ``A short story in prose.'' - ``Prosing on about female beauty.''

\textbf{Irascible}: Having or showing a tendency to be easily angered. ``An irascible man.''

\textbf{Simulacrum}: An image or representation of someone or something / bad imitation. ``A small-scale simulacrum of a skyscraper.''

\textbf{Phantasmagorical}: Illusive, unreal. ``Those suffering from malaria fall into a feverish sleep, their world a whirligig of phantasmagoria.''

\textbf{Discreet}: Careful and circumspect in one's speech or actions, especially in order to avoid causing offense or to gain an advantage. ``We made some discreet inquiries.''

\textbf{Circumspect}: wary and unwilling to take risks. ``the officials were very circumspect in their statements.''

\textbf{Circumvent}: cleverly find a way out of one's duties or obligations ``One way of circumventing the GRE is to apply to a grad school that does not require GRE scores.''

\textbf{Circumscribe}: restrict or confine. ``their movements were strictly monitored and circumscribed.''

\textbf{Besmirch}: Damage the reputation of (someone or something) in the opinion of others. ``He had besmirched the good name of his family.''

\textbf{Onerous}: (of a task or responsibility) involving a great deal of effort, trouble, or difficulty. ``This is an onerous task...''
 
\textbf{Arduous}: involving or requiring strenuous effort; difficult and tiring. ``an arduous journey.''
 
\textbf{Grueling}: extremely difficult and tiring. ``a grueling schedule.''
 
\textbf{Delineate}: Make clearer. ``Let me delineate some points.''

\textbf{Limpid}: having clarity in terms of expression. ``Her limpid prose made even the most recondite subjects accessible to all.''

\textbf{Knack}: an acquired or natural skill at performing a task. ``she got the knack of it in the end.''

\textbf{Aptitude}: A natural ability to do something. ``He has an aptitude for basketball.''

\textbf{Cognizant}: Being aware. ``statesmen must be cognizant of the political boundaries within which they work.''

\textbf{Overt}: done or shown openly; plainly or readily apparent, not secret or hidden. ``an overt act of aggression.''

\textbf{Conspicuous}: Without any attempt at concealment, completely obvious. ``he was very thin, with a conspicuous Adam's apple.''

\textbf{Secluded}: hidden from view; away from people. ``the gardens are quiet and secluded.''

\textbf{Complacent}: showing smug or uncritical satisfaction with oneself or one's achievements. ``Most of the time he simply can't be bothered with it because he truly is lazy and complacent.''

\textbf{Conciliate}: stop (someone) from being angry or discontented; placate; pacify; make peace with. ``His opponents believed his gesture to be conciliatory, yet as soon as they put down their weapons, he unsheathed a hidden sword.''

\textbf{Maudlin}: self-pityingly or tearfully sentimental, often through drunkenness. ``the drink made her maudlin.''

\textbf{Mawkish}: Sentimental in an exaggerated way, to the point that it is disgusting. ``The film was incredibly mawkish...''

\textbf{Histrionic}: overly theatrical or melodramatic in character or style. ``a histrionic outburst.''

\textbf{Predilection}: Strong liking. ``Monte had a predilection for fine things in life.''

\textbf{Querulous}: Habitually complaining (surekli yakinan, yakinmayi huy edinmis). ``She is a querulous old woman.''

\textbf{Inveterate}: Habitual. ``he was an inveterate gambler.''

\textbf{Gall}: The trait of being rude and impertinent. / feeling of deep and bitter anger. ``The speeding car had the gall to switch five lanes at once.'' - ``In an act of gall, Leah sent compromising photos to her.''

\textbf{Uncompromising}: showing an unwillingness to make concessions to others, especially by changing one's ways or opinions. ``Levein's opinions, honest and uncompromising , have been a hallmark of his reign.''

\textbf{Impertinent}: Not showing respect, rude. ``My boy is quite impertinent.''

\textbf{Pertinent}: relevant or applicable to a particular matter; apposite. ``she asked me a lot of very pertinent questions.''

\textbf{Entrenched}: (of an attitude, habit, or belief) firmly established and difficult or unlikely to change; ingrained. ``Most of our habits are so entrenched that it is difficult for us to change.''

\textbf{Reticent}: Disinclined to talk, not revealing one's thoughts. ``When asked about her father, Helen lost her outward enthusiasm and became rather reticent.''

\textbf{Volubility}: The quality of talking or writing easily and continuously. ``Lack of volubility is bad for a debate.''

\textbf{Gossamer}: Characterized by unusual lightness and delicacy. ``The gossamer wings of a butterfly.''

\textbf{Slender}: thin and attractive. ``Slender man is very attractive.''

\textbf{Embellish}: make (something) more attractive by the addition of decorative details or features. ``blue silk embellished with golden embroidery.''

\textbf{Mellifluous}: (of a voice or words) sweet or musical; pleasant to hear. ``the voice was mellifluous and smooth.''

\textbf{Beatific}: blissfully happy. ``A beatific smile.''

\textbf{Serene}: calm, peaceful, and untroubled; tranquil. ``her eyes were closed and she looked very serene.''

\textbf{Lull}: a temporary interval of quiet or lack of activity. ``for two days there had been a lull in the fighting.''

\textbf{Halcyon}: denoting a period of time in the past that was idyllically (extremely happy, peaceful, or picturesque, pastoral) happy and peaceful. ``The first decade after WWI was a halcyon period in America with new-found wealth and rapidly improving technology.''

\textbf{Thwart}: prevent (someone) from accomplishing something. ``he never did anything to thwart his father.''

\textbf{Preclude}: prevent from happening; make impossible. ``the secret nature of his work precluded official recognition.''

\textbf{Assuage}: make (an unpleasant feeling) less intense. ``Her fear that the new college would be filled with unknown faces was assuaged when she recognized her childhood friend standing in line.''

\textbf{Mitigate}: Make less severe. ``The pain medications mitigated my headache.''

\textbf{Exacerbate}: Make worse. ``Her sleeplessness exacerbated her cold.''

\textbf{Compound}: make (something bad) worse; intensify the negative aspects of. ``I compounded the problem by trying to make wrong things right.''

\textbf{Buttress}: increase the strength of or justification for; reinforce. ``authority was buttressed by religious belief.''

\textbf{Wax}: increase gradually. ``The moon waxed full, looming huge upon the speckled expanse.''

\textbf{Impede}: delay or prevent (someone or something) by obstructing them; hinder. ``the sap causes swelling that can impede breathing.''

\textbf{Hobble}: to hold back the progress of something. ``Bad weather has hobbled rescue efforts, making it difficult for crews to find bodies in the wreckage.''

\textbf{Hamper}: hinder or impede the movement or progress of. ``their work is hampered by lack of funds.''

\textbf{Stymie}: prevent or hinder the progress of. ``the changes must not be allowed to stymie new medical treatments.''

\textbf{Hamstrung}: Made ineffective or powerless. ``The FBI has made so many restrictions on the local police that they are absolutely hamstrung, unable to accomplish anything.''

\textbf{Doughty}: brave; bold; courageous. ``I enjoy films in which a doughty group comes together to battle a force of evil.''

\textbf{Audacious}: Willing to be bold in social situations and take risks. / Showing impudent lack of respect.  ``The persistently audacious are helped along by a fearless temperament.''

\textbf{Effrontery}: insolent or impertinent behavior, audacious (even arrogant) behavior that you have no right to. ``The skateboarders acted with effrontery, skating through the church grounds and spray-painting signs warning trespassers.''

\textbf{Foolhardy}: recklessly bold or rash. ``it would be foolhardy to go into the scheme without support.''

\textbf{Temerity}: excessive confidence or boldness; audacity. ``No child has the temerity to go in the rundown house at the end of the street and see if it is haunted.''

\textbf{Impudent}: not showing due respect for another person; impertinent. ``Impudent teenagers all around.''

\textbf{Temperament}: a person's or animal's nature, especially as it permanently affects their behavior. ``she had an artistic temperament.''

\textbf{Temperance}: the trait of avoiding excesses, moderation. (ölçülülük). ``Welles wasn't known for his temperance--he usually ate enough for two and drank enough for three.''

\textbf{Extant}: (especially of a document) still in existence; surviving. ``the original manuscript is no longer extant.''

\textbf{Mercurial}: (of a person) subject to sudden or unpredictable changes of mood or mind. ``his mercurial temperament.''

\textbf{Judicious}: having, showing, or done with good judgment or sense. ``The efficient and judicious use of pesticides.''

\textbf{Sagacious}: having or showing keen mental discernment and good judgment; shrewd; having good judgment and acute insight. ``Steve Jobs is surely one of the most sagacious CEOs, making Apple one of the most recognizable and valuable companies in the world.''

\textbf{Perspicacious}: having a ready insight into and understanding of things, acute insightful, wise. ``Many modern observers regard Eisenhower as perspicacious, particularly in his accurate prediction of the growth of the military.''

\textbf{Percipient}: (of a person) having a good understanding of things; perceptive. ``he is a percipient interpreter of the public mood.''

\textbf{Imprudent}: Not wise. ``Hitler, like Napoleon, made the imprudent move of invading Russia in winter, suffering even more casualties than Napoleon had.''

\textbf{Impartial}: Fair, just. ``Judges are impartial.''

\textbf{Evenhanded}: fair and impartial in treatment or judgment. ``an even-handed approach.''

\textbf{Equitable}: fair and impartial. ``an equitable balance of power.''

\textbf{Recondite}: (of a subject or knowledge) little known; abstruse. ``I found Ulysses recondite and never finished the book.''

\textbf{Engender}: Give rise to. ``Treaty of Versailles was so severe that it engendered deep hatred and resentment in the German people.''

\textbf{Resentment}: Bitter indignation at having been treated unfairly. ``I displayed my resentment at my demotion.''

\textbf{Resent}: feel bitterness or indignation at (a circumstance, action, or person). ``she resented the fact that I had children.''

\textbf{Indignation}: Anger or annoyance provoked by what is perceived as unfair treatment. ``the letter filled Lucy with indignation.''

\textbf{Indignant}: feeling or showing anger or annoyance at what is perceived as unfair treatment. ``he was indignant at being the object of suspicion.''

\textbf{Incense}: Make (someone) very angry. ``she was incensed by the accusations.''

\textbf{Cavalier}: showing a lack of proper concern; offhand. ``But now his wife has spoken of her bitterness at the cavalier way she believes she was told their 21-year marriage was over.''

\textbf{Chivalrous}: being attentive to women like an ideal knight. ``Medieval tales are full of stories of chivalry, in which a young knight must commit deeds of heroism to win the hand of a fair maiden.''

\textbf{Deferential}: Showing respect. ``If you ever have the chance to meet the president, stand up straight and be deferential.''

\textbf{Gainsay}: Deny or contradict; speak against or oppose. ``I can't gainsay a single piece of evidence James has presented, but I still don't trust his conclusion.''

\textbf{Buck}: oppose or resist (something that seems oppressive or inevitable). ``the shares bucked the market trend.''

\textbf{Defray}: to help pay the cost of, either in part or full. ``In order for Sean to attend the prestigious college, his generous uncle helped defray the excessive tuition with a monthly donation.''

\textbf{Gentrification}:  the process of making a person or activity more refined or polite. ``football has undergone gentrification.''

\textbf{Bucolic}: of or relating to the pleasant aspects of the countryside and country life. ``the church is lovely for its bucolic setting.''

\textbf{Rustic}: (kirsal) / constructed or made in a plain and simple fashion, in particular. ``The furniture has the rough rustic feel you can only get from hand crafting and is reminiscent of old Morocco.'' 

\textbf{Pastoral}: (especially of land or a farm) used for or related to the keeping or grazing of sheep or cattle. ``scattered pastoral farms.''

\textbf{Apathetic}: showing or feeling no interest, enthusiasm, or concern. ``Indifferent, apathetic , she has no reason to think her life will improve.''

\textbf{Decorous}: in keeping with good taste and propriety; polite and restrained. ``dancing with decorous space between partners.''

\textbf{Decorum}: behavior in keeping with good taste and propriety. ``you exhibit remarkable modesty and decorum.''

\textbf{Pedant}: a person who is excessively concerned with minor details and rules or with displaying academic learning. ``The intrusive comma changes the sense, and gives the dedicated pedant a linguistic heart attack.''

\textbf{Punctilious}: showing great attention to detail or correct behavior. ``he was punctilious in providing every amenity for his guests.''

\textbf{Exhaustive}: examining, including, or considering all elements or aspects; fully comprehensive. ``she has undergone exhaustive tests since becoming ill.''

\textbf{Intrusive}: causing disruption or annoyance through being unwelcome or uninvited. (davetsiz). ``that was an intrusive question.''

\textbf{Germane}: relevant to a subject under consideration. ``That is not germane to our question.''

\textbf{Apposite}: apt (appropriate) in the circumstances or in relation to something. ``An apposite question.''

\textbf{Becoming}: (especially of clothing) flattering a person's appearance; suitable, appropriate. ``what a becoming dress!''

\textbf{Immaterial}: Not relevant. ``The judge found the defendant’s comments immaterial to the trial.''

\textbf{Provident}: making or indicative of timely preparation for the future. ``In a move that hardly could be described as provident, Bert spent his entire savings on a luxurious cruise, knowing that other bills would come due a couple months later.''

\textbf{Omniscient}: Knowing everything. ``Omniscient entity.''

\textbf{Prescience}: the fact of knowing something before it takes place; foreknowledge. ``with extraordinary prescience, Jung actually predicted the Nazi eruption.'' 

\textbf{Prognostication}: the action of foretelling or prophesying future events. ``an unprecedented amount of soul-searching and prognostication.''

\textbf{Innocuous}: not harmful or offensive. ``It was an innocuous question.''

\textbf{Insidious}: proceeding in a gradual, subtle way, but with harmful effects. (sinsi). ``Sexually transmitted diseases can be insidious and sometimes without symptoms.''

\textbf{Pernicious}: exceedingly harmful; working or spreading in a hidden and injurious way. ``The most successful viruses are pernicious: an infected person may feel perfectly healthy for several months while incubating and spreading the virus.''

\textbf{Adverse}: preventing success or development; harmful; unfavorable. ``taxes are having an adverse effect on production.''

\textbf{Deleterious}: harmful to living things. ``The BP oil spill in the Gulf of Mexico was deleterious to the fishing industry in the southern states.''

\textbf{Prodigious}: remarkably or impressively great in extent, size, or degree. ``the stove consumed a prodigious amount of fuel.''

\textbf{Garrulous}: excessively talkative, especially on trivial matters. ``Polonius is portrayed as a foolish, garrulous old man.''

\textbf{Precocious}: (of a child) having developed certain abilities or proclivities at an earlier age than usual. ``Any display of precocious talent - or even average ability - mysteriously finds its way into every conversation.''

\textbf{Precarious}: not securely held or in position; dangerously likely to fall or collapse; dependent on chance, uncertain. ``A precarious ladder.''; ``People smoke to relax and forget their cares, but ironically, in terms of health risks, smoking is far more precarious than either mountain-climbing or skydiving.''

\textbf{Catholic}: (especially of a person's tastes) including a wide variety of things; all-embracing. ``(especially of a person's tastes) including a wide variety of things; all-embracing.''

\textbf{Eclectic}: Comprised of a variety of styles. ``I have an eclectic music taste, if you were to look at my song list in lexicographical order, at letter M you would see Mozart, Megadeth and Michael Jackson!''

\textbf{Dwindle}: to become much less or many fewer. ``Due to repeated overfishing, the numbers of snow crabs have severely dwindled in recent years.''

\textbf{Iota}: Extremely small amount. ``nothing she said seemed to make an iota of difference.''

\textbf{Minute}: Extremely small. ``Unfortunately for her students, the professor noticed even the most minute mistakes in essays that she graded.''

\textbf{Diminutive}: extremely or unusually small / a smaller or shorter thing, in particular. ``a diminutive figure dressed in black.''

\textbf{Pittance}: a very small or inadequate amount of money paid to someone as an allowance or wage. ``With a pittance of a salary, how could they be enthused to become proactive people?''

\textbf{Smattering}: a slight superficial knowledge of a language or subject. ``Edward had only a smattering of Spanish.'' - ``A smattering of several subjects does not necessarily make you eclectic nor dilettante.''

\textbf{Picayune}: petty; worthless. ``the picayune squabbling of party politicians.''

\textbf{Petty}: of little importance; trivial. ``the petty divisions of party politics.''

\textbf{Rife}: (especially of something undesirable or harmful) of common occurrence; widespread. ``Never believe anything you hear in a politician's speech; they are rife with inaccuracies and even lies.''

\textbf{Raft}: A large amount of something. ``Despite a raft of city ordinances passed by an overzealous council, noise pollution continued unabated in the megalopolis.''

\textbf{Abstain}: restrain oneself from doing or enjoying something. ``Abstain from chocolate.''

\textbf{Staunch}: loyal and committed in attitude. ``I'm no longer a staunch supporter of the movement.''

\textbf{Spurious}: not being what it purports to be; false or fake. ``separating authentic and spurious claims''.

\textbf{Belie}: (of an appearance) fail to give a true notion or impression of (something); disguise or contradict. ``his lively alert manner belied his years.''

\textbf{Authentic}: of undisputed origin; genuine. ``the letter is now accepted as an authentic document.''

\textbf{Vilify}: speak or write about in an abusively disparaging manner. ``he has been vilified in the press.''

\textbf{Disparaging}: expressing the opinion that something is of little worth; derogatory. ``A disparaging speech.''

\textbf{Contrive}: create or bring about (an object or a situation) by deliberate use of skill and artifice (cunning tricks used to deceive others). ``his opponents contrived a crisis.''

\textbf{Concede}: Acknowledge defeat. / admit wrongdoing / surrender to the physical control of another. ``I concede, you win!''

\textbf{Platitude}: A title or obvious remark. ``This year more than ever, the hack politician's laziest platitude is true: ``This election is about the future.''''

\textbf{Belligerent}: Characteristic of one eager to fight. ``Belligerent hooligans.''

\textbf{Truculent}: eager or quick to argue or fight, aggressively defiant. ``his days of truculent defiance were over.''

\textbf{Firebrand}: a person who is passionate about a particular cause, typically inciting change and taking radical action. ``Freddie is a firebrand: every time he walks into the office, he winds up at the center of heated argument.''

\textbf{Propensity}: an inclination or natural tendency to behave in a particular way. ``a propensity for violence.''

\textbf{Choleric}: prone to outbursts of temper, easily angered. ``While a brilliant lecturer, Mr. Dawson came across as choleric and unapproachable—very rarely did students come to his office hours.''

\textbf{Oblique}: not straightforward; indirect. ``Herbert never explicitly revealed anything negative about Tom's past, but at times he would obliquely suggest that Tom was not as innocent as he seemed.''

\textbf{Hazy}: covered by a haze (a state of mental obscurity or confusion); unclear. ``A typical ten-year-old has only a hazy idea of how to manage money and how important it is to do so.''

\textbf{Equivocal}: open to more than one interpretation; ambiguous.  ``the equivocal nature of her remarks.''

\textbf{Unequivocal}: leaving no doubt; unambiguous. ``An unequivocal answer.''

\textbf{Fallacious}: Based on a mistaken belief, prone to error. ``Fallacious arguments.''

\textbf{Mollify}: Make someone angry less angry; placate. ``nature reserves were set up around the power stations to mollify local conservationists.''

\textbf{Rankle}: gnaw into; make resentful or angry, get under one's skin. ``His constant whistling would rankle her, sometimes causing her to leave in a huff.''

\textbf{Quotidian}: Of or occurring every day; daily. ``Some quotidian concerns: exercising and eating regularly.''

\textbf{Rudimentary}: involving or limited to basic principles. (ilkel). ``I would love to be able to present a fully polished proposal to the board, but right now, our plans for the product are still in the most rudimentary stages.''

\textbf{Haughty}: arrogantly superior and disdainful. ``We have a haughty manager...''

\textbf{Hauteur}: haughtiness of manner; disdainful pride. ``As soon as she won the lottery, Alice began displaying a hauteur to her friends, calling them dirty-clothed peasants behind their backs.''

\textbf{Forlorn}: (of an aim or endeavor) unlikely to succeed or be fulfilled; hopeless. / abandoned, lonely. ``a forlorn attempt to escape.'' - ``Forlorn figures at bus stops.''

\textbf{Languid}: not inclined towards physical exertion or effort; slow and relaxed. ``My weekends are languid, mostly Netflix.''

\textbf{Strenuous}: requiring a lot of physical effort. ``One of the most strenuous races on earth, the Ironman Triathlon requires swimming 2.4 miles, biking 112 miles, and then running a marathon.''

\textbf{Ameliorate}: Make something bad better. ``I hope to ameliorate poverty.''

\textbf{Complaisant}: showing a cheerful willingness to do favors for others. ``On her first day at the job, Annie was complaisant, fulfilling every request of her new employer and anticipating future requests.''

\textbf{Solicitous}: characterized by or showing interest or concern. ``she was always solicitous about the welfare of her students.''

\textbf{Disaffected}: dissatisfied with the people in authority and no longer willing to support them. ``I am bewildered by how Turkish people are still not disaffected considering the economic situation.''

\textbf{Itinerant}: Traveling from place to place to work. ``My father's itinerant work caused me to change schools very often.''

\textbf{Jingoism}: Fanatical patriotism. ``Jingoism in North Korea.''

\textbf{Haphazard}: lacking any obvious principle of organization, random. ``the kitchen drawers contained a haphazard collection of silver souvenir spoons.''

\textbf{Winsome}: charming in a childlike or naive way. ``She was winsome by nature, and many people were drawn to this free and playful spirit.''

\textbf{Jovial}: cheerful and friendly. ``she was in a jovial mood.''

\textbf{Sanguine}: Cheerful; optimistic; blood-red. ``he is sanguine about prospects for the global economy.'' - ``Instances later, she was a beautiful young maiden with sanguine hair and a scarlet dress.''

\textbf{Aplomb}: self-confidence or assurance, especially when in a demanding situation. ``Nancy acted with aplomb during dangerous situations--she once calmly climbed up an oak tree to save a cat.''

\textbf{Sangfroid}: composure or coolness, sometimes excessive, as shown in danger or under trying circumstances. ``Offering the most welcoming stage for the talented, the city with equal sangfroid accepts the misery of millions who fail to flourish.''

\textbf{Unflappable}: having or showing calmness in a crisis. ``Malati, a sedate old female, was a placid soul, unflappable even in a crisis.''

\textbf{Nonchalant}: (of a person or manner) feeling or appearing casually calm and relaxed; not displaying anxiety, interest, or enthusiasm. ``she gave a nonchalant shrug.''

\textbf{Convivial}: (of an atmosphere or event) friendly, lively, and enjoyable. ``It was a lively, convivial atmosphere - gone but not forgotten.''

\textbf{Dreary}: dull, bleak, and lifeless; depressing. ``the dreary routine of working, eating, and trying to sleep.''

\textbf{Saturnine}: Morose, gloomy. ``Deprived of sunlight, humans become saturnine.''

\textbf{Sullen}: bad-tempered and sulky; gloomy. ``When I mentioned this, he lapsed back into the sullens.''

\textbf{Phlegmatic}: (of a person) having an unemotional and stolidly calm disposition. ``Arnold is truly noble, remaining reserved until an issue of significance arises, but Walter is simply phlegmatic: he doesn't have the energy or inclination to care about anything.''

\textbf{Immutable}: Not able to be changed. ``Tax is an immutable law of the land.''

\textbf{Commensurate}: corresponding in size or degree; in proportion. ``salary will be commensurate with experience.''

\textbf{Juxtapose}: Place side by side for contrast. ``The appeal of her paintings comes from a classical style which is juxtaposed with modern themes.''

\textbf{Tantamount}: equivalent in seriousness to; virtually the same as. ``the resignations were tantamount to an admission of guilt.''

\textbf{Supplant}: Take the place or move into the position of. ``Cell phone has supplanted the traditional phone.''

\textbf{Panache}: Distinctive and showy elegance. ``I did it with panache...''

\textbf{Jubilant}: Feeling or expressing great happiness and triumph. ``I was jubilant to receive a perfect score on the GRE.''

\textbf{Heretic}: A person who holds unorthodox opinions in any field. ``Bretz was called a dunce and a heretic , but over time his work became widely accepted.''

\textbf{Travesty}: A false, absurd, or distorted representation of something. ``Michael has betrayed the family by travestying them in his plays.''

\textbf{Graft}: practices, especially bribery, used to secure illicit gains in politics or business; corruption. ``sweeping measures to curb official graft.''

\textbf{Subterfuge}: deceit used in order to achieve one's goal. ``Finally deciding to abandon all subterfuge, Arthur revealed to Cindy everything about his secret affair over the past two years.''

\textbf{Hoodwink}: to deceive or trick someone. ``Someone tried to hoodwink Marty with an email telling him that his uncle had just passed away, and to collect the inheritance he should send his credit card information.''

\textbf{Dissemble}: conceal one's true motives, usually through deceit. ``To get close to the senator, the assassin dissembled his intentions, convincing many people that he was a reporter for a well-known newspaper.''

\textbf{Surreptitious}: kept secret, especially because it would not be approved of. ``they carried on a surreptitious affair.''

\textbf{Ascetic}: A person who practices severe self-discipline and abstention. ``an ascetic life of prayer, fasting, and manual labor.''

\textbf{Unstinting}: given or giving without restraint; unsparing; generous. ``Helen is unstinting with her time, often spending hours at the house of a sick friend.''

\textbf{Magnanimous}: Very generous or forgiving, especially toward a rival or someone less powerful than oneself. ``He was a great sportsman: in defeat he was complimentary and in victory he was magnanimous.''

\textbf{Eminent}: Standing above others in quality or position. ``Shakespeare is an eminent author in the English language, but I find his writing uninteresting and melodramatic.''

\textbf{Pragmatic}: Dealing with things sensibly and realistically in a way that is based on practical rather than theoretical considerations. ``A pragmatic approach to politics.''

\textbf{Pundit}: An expert in a particular subject or field who is frequently called on to give opinions about it to the public. ``a globe-trotting financial pundit.''

\textbf{Ingratiate}: Bring oneself into favor with someone by flattering or trying to please them. ``My cat ingratiates every stranger it meets.''

\textbf{Acquaintance}: A person's knowledge or experience of something. / (tanidik). ``The students had little acquaintance with the language.''

\textbf{Ignoble}: not honorable in character or purpose. ``ignoble feelings of intense jealousy.''

\textbf{Munificent}: Very generous. ``Uncle Charley was known for his munificence, giving all seven of his nephews lavish Christmas presents each year.''

\textbf{Largesse}: generosity in bestowing money or gifts upon others. ``presumably public money is not dispensed with such largesse to anyone else.''

\textbf{Reprobate}: unprincipled (often used as a humorous or affectionate reproach). ``a long-missed old reprobate drinking comrade.''

\textbf{Unscrupulous}:  without scruples or principles. ``In the courtroom, the lawyer was unscrupulous, using every manner of deceit and manipulation to secure a victory for himself.''

\textbf{Scrupulous}: (of a person or process) diligent, thorough, and extremely attentive to details. ``the research has been carried out with scrupulous attention to detail.''

\textbf{Swivel}: to spin or turn around. ``Lily's chair squeaked when it rotated, which drove her coworkers crazy due to her habit of swiveling while thinking.''

\textbf{Tortuous}: full of twists and turns. ``the route is remote and tortuous.''

\textbf{Persevere}: continue in a course of action even in the face of difficulty or with little or no prospect of success. ``his family persevered with his treatment.''

\textbf{Checkered}: marked by disreputable or unfortunate happenings. ``One of the people in your report had a checkered past.''

\textbf{Connive}: secretly allow (something considered immoral, illegal, wrong, or harmful) to occur. ``you have it in your power to connive at my escape.''

\textbf{Intransigent}: unwilling or refusing to change one's views or to agree about something. ``Despite many calls for mercy, the judge remained intransigent, citing strict legal precedence.''

\textbf{Adamant}: Refusing to change one's mind. ``Civil rights icon Rosa Parks will forever be remembered for adamantly refusing to give up her seat on a public bus--even after the bus driver insisted, she remained rooted in place.''

\textbf{Burgeon}: Grow and flourish. ``manufacturers are keen to cash in on the burgeoning demand.'' 

\textbf{Incorrigible}: (of a person or their tendencies) not able to be corrected, improved, or reformed. ``she's an incorrigible flirt.''

\textbf{Bumbling}: acting in a confused or ineffectual way; incompetent. ``Possibly torpor, she was bumbling and did not make sense.''

\textbf{Maladroit}: ineffective or bungling; clumsy. ``A maladroit waitress.''

\textbf{Malady}: A disease or sickness; ailment (a small sickness or cause of physical discomfort). ``an incurable malady''

\textbf{Admonitory}: giving or conveying a warning or reprimand. ``the sergeant lifted an admonitory finger.''

\textbf{Admonish}: Warn or reprimand someone firmly. ``she admonished me for appearing at breakfast unshaven.''

\textbf{Implacable}: unable to be placated. ``Our coach is always implacable.''

\textbf{Redress}: remedy or set right (an undesirable or unfair situation). ``Barry's redress for forgetting his wife's birthday two years in a row was surprising her with a trip to Tahiti.''

\textbf{Extenuate}: make (guilt or an offense) seem less serious or more forgivable. ``there were extenuating circumstances that caused me to say the things I did.''

\textbf{Exonerate}: pronounce not guilty of criminal charges. ``The document clearly indicated that Nick was out of the state at the time of the crime, and so served to exonerate him of any charges.''

\textbf{Indict}: to formally charge or accuse of wrong-doing. ``The bank robber was indicted on several major charges, including possession of a firearm.''

\textbf{Vindictive}: having or showing a strong or unreasoning desire for revenge. ``Though the other girl had only lightly poked fun of Vanessa's choice in attire, Vanessa was so vindictive that she waited for an entire semester to get the perfect revenge.''

\textbf{Destitute}: without the basic necessities of life. ``the charity cares for destitute children.'' 

\textbf{Wanting}: lacking in a certain required or necessary quality. ``they weren't wanting in confidence.''

\textbf{Bereft}: deprived of or lacking something, especially a non-material asset / unhappy in love; suffering from unrequited love. ``her room was stark and bereft of color.'' / ``After 64 years of marriage, William was bereft after the death of his wife.''

\textbf{Modicum}: a small quantity of a particular thing, especially something considered desirable or valuable. ``his statement had more than a modicum of truth.''

\textbf{Paucity}: the presence of something only in small or insufficient quantities or amounts; scarcity. ``a paucity of information.''

\textbf{Desecrate}: treat (a sacred place or thing) with violent disrespect; violate. ``more than 300 graves were desecrated.''

\textbf{Myopic}: lacking foresight or imagination. ``The company ultimately went out of business because the myopic managers couldn't predict the changes in their industry.''

\textbf{Hackneyed}: lacking significance through having been overused. ``Cheryl rolled her eyes when she heard the lecturer's hackneyed advice to "be true to yourself."''

\textbf{Acerbic}: (especially of a comment or style of speaking) sharp and forthright, harsh in tone. ``I nearly offered my usual acerbic response of ‘And your point is, exactly?’''

\textbf{Vitriolic}: filled with bitter criticism or malice, harsh or corrosive in tone. ``vitriolic attacks on the politicians.''

\textbf{Abstruse}: Difficult to understand. ``An abstruse lecture.''

\textbf{Expound}: present and explain (a theory or idea) systematically and in detail. ``he was expounding a powerful argument.''

\textbf{Espouse}: to adopt or support an idea or cause. ``he turned his back on the modernism he had espoused in his youth.''

\textbf{Intrepid}: Fearless. ``An intrepid person.''

\textbf{Sycophant}: a person who acts obsequiously toward someone important in order to gain advantage, flatterer. ``An assortment of hatchet men, opportunists and sycophants gained access to the levers of power.''

\textbf{Obsequious}: obedient or attentive to an excessive or servile degree. ``The obsequious waiter did not give the couple a moment's peace all through the meal.''

\textbf{Inimical}: hostile (usually describes conditions or environments). ``Venus, with a surface temperature that would turn rubber to liquid, is inimical to any form of life.''

\textbf{Propitious}: giving or indicating a good chance of success; favorable. ``The child's heartbeat is still weak, but I am seeing many propitious signs and I think that she may be healing.''

\textbf{Propitiate}: win or regain the favor of (a god, spirit, or person) by doing something that pleases them. ``the pagans thought it was important to propitiate the gods with sacrifices.''

\textbf{Antithetical}: sharply contrasted in character or purpose. ``people whose religious beliefs are antithetical to mine.''

\textbf{Repudiate}: refuse to accept or be associated with. ``she has repudiated policies associated with previous party leaders.''

\textbf{Hegemony}: leadership or dominance, especially by one country or social group over others. ``Germany was united under Prussian hegemony after 1871.''

\textbf{Myriad}: A large indefinite number. ``There are a myriad of internet sites hawking pills that claim to boost energy for hours on end.''

\textbf{Precipitous}: (of an action) done suddenly and without careful consideration. ``precipitous intervention.''

\textbf{Precipitate}: hasty or rash; cause (an event or situation, typically one that is bad or undesirable) to happen suddenly, unexpectedly, or prematurely. ``I must apologize for my staff—their actions were precipitate.'' - ``the incident precipitated a political crisis.''

\textbf{Expansive}: (of a person or their manner) open, demonstrative, and communicative. ``she felt expansive and inclined to talk.''

\textbf{Demonstrative}: (of a person) tending to show feelings, especially of affection, openly. ``I'm not a very demonstrative person, having always been taught that emotion leads to weakness, so I was more than a little embarrassed.''

\textbf{Gauche}: lacking ease or grace; unsophisticated and socially awkward. ``Do you think this is a bit loud, you know, gauche, for someone in my condition?''

\textbf{Subvert}: undermine the power and authority of (an established system or institution). ``an attempt to subvert democratic government.''

\textbf{Inscrutable}: Not easily understood; unfathomable. ``His speech was so dense and confusing that many in the audience found it inscrutable.''

\textbf{Prolific}: Very productive. ``The prolific scorer netted a hat-trick in this win over Brooklands.''

\textbf{Pedestrian}: lacking inspiration or excitement; dull. ``disenchantment with their present, pedestrian lives.''

\textbf{Prosaic}: having the style or diction of prose; lacking poetic beauty. ``prosaic language can't convey the experience.''

\textbf{Insipid}: lacking flavor, dull and uninteresting. ``I'm not interested in stupid, insipid men who flower me with ridiculous comments in the hope that I'll fall madly in love with them.''

\textbf{Jejune}: lacking flavor, dull / immature; childish. ``Although many top chefs have secured culinary foam's popularity in haute cuisine, Waters criticizes it for being jejune and unfilling.'' - ``Her boss further cemented his reputation for being jejune after throwing a fit when the water cooler wasn't refilled.''

\textbf{Disenchantment}: a feeling of disappointment about someone or something you previously respected or admired; disillusionment. ``growing disenchantment with the leadership.''

\textbf{Lionize}: assign great social importance to. ``Students in the U.S. learn to lionize Jefferson, Franklin, and Washington because they are the founding fathers of the nation.''

\textbf{Foment}: try to stir up public opinion. ``After having his pay cut, Phil spread vicious rumors about his boss, hoping to foment a general feeling of discontent.''

\textbf{Discontent}: lack of contentment; dissatisfaction with one's circumstances. ``popular discontent with the system had been general for several years.''

\textbf{Contentious}: likely to argue. ``Since old grandpa Harry became very contentious during the summer when only reruns were on T.V., the grandkids learned to hide from him at every opportunity.''

\textbf{Poignant}: emotionally touching, evoking a keen sense of sadness or regret. ``a poignant reminder of the passing of time.''

\textbf{Elegiac}: (especially of a work of art) having a mournful quality; expressing sorrow. ``Few can listen to the elegiac opening bars of the Moonlight sonata without feeling the urge to cry.''

\textbf{Bemoan}: express discontent or sorrow over (something). ``single women bemoaning the absence of men.''

\textbf{Lachrymose}: tearful or given to weeping, showing sorrow. ``Lachrymose and depressed, Alexei Alexandrovich walked two miles home in the rain after learning that his wife was having an affair.''

\textbf{Dolorous}: feeling or expressing great sorrow or distress. ``Far from being dour and dolorous, one of the clearest fruits of grace is a childlike joy.''

\textbf{Specious}: superficially plausible, but actually wrong. ``Almost every image on TV is specious and not to be trusted.''

\textbf{Dubious}: hesitating or doubting, suspicious and possibly wrong. ``I saw a carton of milk with an expiration date so dubious I decided to buy it for fun: it has a date one year from today printed on it.''

\textbf{Askance}: with an attitude or look of suspicion or disapproval. ``The police looked askance on the man with no injuries and clean suit who supposedly got out of a car crash.''

\textbf{Ignominious}: deserving or causing public disgrace or shame. ``An ignominious loss.''

\textbf{Elicit}: evoke or draw out (a response, answer, or fact) from someone in reaction to one's own actions or questions. ``they invariably elicit exclamations of approval from guests.''

\textbf{Edifying}: providing moral or intellectual instruction. ``edifying literature.''

\textbf{Dilatory}: Wasting time; slow to act. ``he had been dilatory in appointing a solicitor.'' - ``My dilatoriness on checking in for military service almost ended up me being awol from military.''

\textbf{Solicitude}: care or concern for someone or something. ``I was touched by his solicitude.''

\textbf{Maverick}: someone who exhibits great independence in thought and action. ``Officer Kelly was a maverick, rarely following police protocols or adopting the conventions for speech common among his fellow officers.''

\textbf{Posit}: Initially, Einstein posited a repulsive force to balance Gravity, but then rejected that idea as a blunder.

\textbf{Appropriate}: to give or take something by force. ``His images have been appropriated by advertisers.''

\textbf{Duress}: Compulsory force or threat. (baski; zorlama). ``confessions extracted under duress.''

\textbf{Decry}: publicly denounce, express strong disapproval of. ``The entire audience erupted in shouts and curses, decrying the penalty card issued by the referee.''

\textbf{Pejorative}: expressing contempt or disapproval. ``Most psychologists object to the pejorative term "shrink", believing that they expand the human mind, not limit it.''
 
\textbf{Mendacity}: Untruthfullness. ``I can forgive her for her mendacity but only because she is a child and is seeing what she can get away with.''

\textbf{Veracious}: Truthful. ``Every word we speak will be veracious.''

\textbf{Craven}: Pathetically cowardly. ``Society does not favor craven people.''

\textbf{Feckless}: Lazy and irresponsible. ``Two years after graduation, Charlie still lived with his parents and had no job, becoming more feckless with each passing day.''

\textbf{Alacrity}: brisk (quick and energetic) and cheerful readiness. ``she accepted the invitation with alacrity.''

\textbf{Timorous}: showing or suffering from nervousness, fear, or a lack of confidence; timid by nature; shy (urkek). ``Since this was her first time debating on stage and before an audience, Di's voice was timorous and quiet for the first 10 minutes.''

\textbf{Grovel}: lie or move abjectly on the ground with one's face downward and show submission or fear. ``Every time Susan comes to the office, Frank grovels as if she were about to fire him.''

\textbf{Quail}: feel or show fear or apprehension. ``she quailed at his heartless words.''

\textbf{Fastidious}: very attentive to and concerned about accuracy and detail. ``Whitney is fastidious about her shoes, arranging them on a shelf in a specific order, each pair evenly spaced.'' - ``My table is that of a fastidious person. Everything is in order and it stays in order no matter what.''

\textbf{Nonplus}: surprise and confuse (someone) so much that they are unsure how to react. ``I was nonplussed by my friends incredulousness towards the fact that earth is round.''

\textbf{Credulous}: having or showing too great a readiness to believe things. ``Children are credulous.''

\textbf{Contrition}: the state of feeling remorseful and penitent. (contrite is the verb form). ``To show contrition for his crime he offered to do community service.''

\textbf{Bolster}: support or strengthen; prop up. ``The fall in interest rates is starting to bolster confidence.''
 
\textbf{Opulence}: great wealth or luxuriousness. ``rooms of spectacular opulence.''
 
\textbf{Stringent}: (of regulations, requirements, or conditions) strict, precise, and exacting. ``California's air pollution guidelines are stringent.'' - ``Most academic journals are stringent on their format rules.''

\textbf{Importune}: ask (someone) pressingly and persistently for or to do something. ``if he were alive now, I should importune him with my questions.''

\textbf{Harried}: feeling strained as a result of having demands persistently made on one; harassed. ``With a team of new hires to train, Martha was constantly harried with little questions and could not focus on her projects.''

\textbf{Harry}: persistently carry out attacks on (an enemy or an enemy's territory). ``He continued to attack, harry and chase every ball and was rewarded late on with a dramatic Golden Goal.''

\textbf{Hector}: talk to (someone) in a bullying way. ``she doesn't hector us about giving up things.''

\textbf{Efficacious}: (typically of something inanimate or abstract) successful in producing a desired or intended result; effective. ``the vaccine has proved both efficacious and safe.''

\textbf{Esoteric}: intended for or likely to be understood by only a small number of people with a specialized knowledge or interest, understood by few. ``esoteric philosophical debates.'' - ``He is a peculiar person with eclectic interest in science, our conversations are usually esoteric and often times I am unable to follow.''

\textbf{Mundane}: Repetitive and boring; ordinary; earthly. ``seeking a way out of his mundane, humdrum existence.'' - ``the boundaries of the mundane world.''

\textbf{Disseminate}: spread or disperse (something, especially information) widely. ``That's our job, is to disseminate information to the public.''

\textbf{Furtive}: attempting to avoid notice or attention, typically because of guilt or a belief that discovery would lead to trouble; secretive. ``While at work, George and his boss Regina felt the need to be as furtive as possible about their romantic relationship.''

\textbf{Appease}: pacify or placate (someone) by acceding to their demands. ``Neville Chamberlain, the British prime minister during WWII, tried to appease Hitler and in doing so sent a clear message: you can walk all over us.''

\textbf{Macabre}: disturbing and horrifying because of involvement with or depiction of death and injury. ``Edgar Allen Poe was considered the master of the macabre.''

\textbf{Urbane}: (of a person, especially a man) suave, courteous, and refined in manner. ``Educated at one of Shanghai's top universities, he's urbane , articulate in English and works in a foreign law firm.''

\textbf{Courteous}: polite, respectful, or considerate in manner. ``But you can at least be polite, courteous and respect the fact that your views are very different to theirs.''

\textbf{Indigenous}: Originating in a certain area. ``The plants and animals indigenous to Australia are notably different from those indigenous to the U.S—one look at a duckbill platypus and you know you’re not dealing with an opossum.''

\textbf{Indigent}: poor, having very little. ``In the so-called Third World, many are indigent and only a privileged few have the resources to enjoy material luxuries.''

\textbf{Lavender}: A kind of flower. (lavanta cicegi). ``Most flowers are lavender , purple, or fluorescent pink.''

\textbf{Inundate}: to flood; overwhelm. ``The newsroom was inundated with false reports that only made it more difficult for the newscasters to provide an objective account of the bank robbery.''

\textbf{Squander}: waste (something, especially money or time) in a reckless and foolish manner. ``entrepreneurs squander their profits on expensive cars.''

\textbf{Spendthrift}: One who spends money extravagantly. ``A spendthrift person.''

\textbf{Profligate}: recklessly extravagant or wasteful in the use of resources. ``profligate consumers of energy.''

\textbf{Dissipate}: squander or spend money frivolously. / to scatter. ``The recent graduates dissipated their earnings on trips to Las Vegas and cruises in Mexico.''

\textbf{Lavish}: sumptuously rich, elaborate, or luxurious. ``a lavish banquet.''

\textbf{Thrifty}: (of a person or their behavior) using money and other resources carefully and not wastefully. ``I'm cheap (or rather thrifty and money smart), but a lot of cultures don't tip.''

\textbf{Prodigal}: spending money or resources freely and recklessly; wastefully extravagant. ``prodigal habits die hard.''

\textbf{Parsimonious}: unwilling to spend money or use resources; stingy or frugal. ``parsimonious New Hampshire voters, who have a phobia about taxes.''

\textbf{Frugal}: sparing or economical with regard to money or food. ``She led a remarkably frugal existence.''

\textbf{Pauper}: a very poor person. ``Disease spread rapidly among the half starved and half clothed paupers.''

\textbf{Penurious}: extremely poor, poverty-stricken, lacking money. ``Truly penurious, Mary had nothing more than a jar full of pennies.''

\textbf{Mendicant}: a pauper who lives by begging. ``Possibly it was sheer vanity and love of easily-won applause that drove him to act out the role of mendicant campus guru.''

\textbf{Tawdry}: showy but cheap and of poor quality. ``tawdry jewelry.''

\textbf{Miser}: a person who doesn't like to spend money (because they are greedy). ``Monte was no miser, but was simply frugal, wisely spending the little that he earned.''

\textbf{Impecunious}: having little or no money. ``a titled but impecunious family.''

\textbf{Exiguity}: the quality of being meager (of something provided or available that is lacking in quantity or quality). ``After two months at sea, the exiguity of the ship's supplies forced them to search for fresh water and food.''

\textbf{Pecuniary}: relating to or involving money. ``The defendant was found guilty and had to serve a period of community service as well as pay pecuniary damages to the client.''

\textbf{Insolvent}: unable to pay debts owed. ``A great number of Turkish companies will become insolvent this year.''

\textbf{Affluent}: Wealthy. ``He is born into an affluent family.''

\textbf{Diffident}: modest or shy because of a lack of self-confidence. ``a diffident youth.''

\textbf{Demure}: Modest and shy. ``The portrait of her in a simple white blouse was sweet and demure.''

\textbf{Demur}: the action or process of objecting to or hesitating over something. ``they accepted this ruling without demur.''

\textbf{Peruse}: To read very carefully. ``I doubt anyone peruses the terms and conditions.''

\textbf{Remiss}: Negligent in ones duty. ``I've been remiss.''

\textbf{Impetus}: the force or energy with which a body moves; a cause or reason to start an action. ``Before John met Amanda, all his shirts were old and filled with holes, but his desire to impress her gave him the impetus to buy all new clothes.''

\textbf{Pretext}: an untruthful reason given for an action. (bahane). ``As the movie started, Karthik moved slowly closer to Sofia under the pretext of getting more comfortable in his seat.''

\textbf{Impetuous}: acting or done quickly and without thought or care. ``Herbert is rarely impetuous, but on the spur (a thing that prompts or encourages someone; an incentive) of the moment, he spent thousands of dollars on a motorcycle today.''

\textbf{Slapdash}: done too hurriedly and carelessly. ``The office building had been constructed in a slapdash manner.''

\textbf{Pine}: To yearn for. ``She pined for her lost love.''

\textbf{Doleful}: filled with or evoking sadness. ``No event is more doleful than the passing of my mother; she was a shining star in my life, and it brings me great sadness to think that she is now gone.''

\textbf{Lugubrious}: looking or sounding sad and dismal (depressing, hopeless), excessively mournful. ``At the funeral, lugubrious songs filled the small church.''

\textbf{Despondent}: extremely sad, in low spirits from loss of hope or courage. ``Fariha was a very strong woman, and she never became despondent no matter how many misfortunes she went through.''

\textbf{Crestfallen}: sad and disappointed. ``he came back empty-handed and crestfallen.''

\textbf{Mulct}: a fine or compulsory payment / defraud or swindle. ``The punishments and penalty for first offense could include a jailhouse term of up to one twelvemonth plus a mulct of up to \$3000.'' - ``The so-called magical diet cure simply ended up mulcting Maria out of hundreds of dollars, but did nothing for her weight.''

\textbf{Dupe}: To trick or swindle / A person who is easily tricked or swindled. ``Once again a get-rich-fast Internet scheme had duped Harold into submitting a \$5,000 check to a sham operation.''

\textbf{Duplicity}: deceitfulness; double-dealing. ``the president was accused of duplicity in his dealings with Congress.''

\textbf{Harangue}: aggressive speech. ``When he finished his lengthy harangue , everyone left, and Lohia wandered over to the nearest paanwallah to ask if Hanif was out yet.''

\textbf{Screed}: a long speech or piece of writing, typically one regarded as tedious, abusive rant. ``There are other misimpressions created by your articles that necessitate a response, but I did not intend for this to become a screed against your writing.''

\textbf{Stipend}: a fixed regular sum paid as a salary or allowance. ``The rise was ‘three per cent - a percentage in line with the rise in stipends for all clergymen,’ he said.''

\textbf{Thoroughgoing}: involving or attending to every detail or aspect of something. ``a thoroughgoing reform of the whole economy.''

\textbf{Hound}: harass, persecute, or pursue relentlessly. ``a tenacious attorney general who had hounded Jimmy Hoffa and other labor bosses.''

\textbf{Ferret}: To search for something persistently. ``Ever the resourceful lexicographer, Fenton was able to ferret out the word origin of highly obscure words.''

\textbf{Jingoist}: a person who thinks their country is always right and who is in favor of aggressive acts against other countries. ``In the days leading up to war, a nation typically breaks up into the two opposing camps: doves, who do their best to avoid war, and jingoists, who are only too eager to wave national flags from their vehicles and vehemently denounce those who do not do the same.''

\textbf{Chauvinist}: a person displaying aggressive or exaggerated patriotism. ``The chauvinist lives on both sides of the political spectrum, outright shunning anybody whose ideas are not consistent with his own.''

\textbf{Calumny}: the making of false and defamatory statements in order to damage someone's reputation; slander. ``Incidentally, he takes her to task for disseminating such calumny.''

\textbf{Slander}: the action or crime of making a false spoken statement damaging to a person's reputation. ``he is suing the TV network for slander.''

\textbf{Ambivalent}: having mixed feelings or contradictory ideas about something or someone. ``some loved her, some hated her, few were ambivalent about her.''

\textbf{Qualify}: be entitled to a particular benefit or privilege by fulfilling a necessary condition / make (a statement or assertion) less absolute; add reservations to; make less severe; to clarify by making less general ``they do not qualify for compensation payments.'' - ``she felt obliged to qualify her first short answer.''

\textbf{Dilapidated}: (of a building or object) in a state of disrepair or ruin as a result of age or neglect. ``The main house has been restored but the gazebo is still dilapidated and unusable..''

\textbf{Enervate}: cause (someone) to feel drained of energy or vitality; weaken.  ``The knowledge of a shared destiny energizes and sustains many of us, enervates and defeats others.''

\textbf{Cogent}: Clear and persuasive. ``A cogent argument will change the minds of even the most skeptical audience, this is prevalent among people with good rhetoric.''

\textbf{Hyperbole}: exaggerated statements or claims not meant to be taken literally. ``In a literary world filled with emotionalism and hyperbole , there are a few guiding stars.''

\textbf{Hodgepodge}: a confused mixture. ``Rob's living room was a hodgepodge of modern furniture and antiques.''

\textbf{Perquisite}: a thing regarded as a special right or privilege enjoyed as a result of one's position. ``Even as the dishwasher at the French restaurant, Josh quickly learned that he had the perquisite of being able to eat terrific food for half the price diners would pay.''

\textbf{Consensus}: an agreement reached by a whole group of people. ``Alicia and her sisters argued for weeks about who would live in the house their parents had left, eventually reaching a consensus to simply sell it.''

\textbf{Junta}: a military or political group that rules a country after taking power by force. ``the country's ruling military junta.''

\textbf{Disparate}: essentially different in kind; not allowing comparison. ``they inhabit disparate worlds of thought.''

\textbf{Miscreant}: a person who behaves badly or in a way that breaks the law. ``Mercifully, just when a major surgical procedure looked inevitable for the miscreant toddler, Mr Scott had a flash of inspiration.''

\textbf{Misogynist}: a person who dislikes, despises, or is strongly prejudiced against women. ``a misogynist attitude.''

\textbf{Misanthrope}: A hater of mankind. ``It's well written and funny, and the author is clearly a misanthrope , which is always a plus in my opinion.''

\textbf{Unconscionable}: unreasonable; unscrupulous; excessive. ``The lawyer’s demands were so unconscionable that rather than pay an exorbitant sum or submit himself to any other inconveniences, the defendant decided to find a new lawyer.''

\textbf{Exorbitant}: (of a price or amount charged) unreasonably high. ``the exorbitant price of tickets.''

\textbf{Vicarious}: experienced in the imagination through the feelings or actions of another person; experienced through a second person. ``I could glean vicarious pleasure from the struggles of my imaginary film friends.''

\textbf{Glean}: extract (information) from various sources, collect information bit by bit. ``So what can we glean from our collection of titles?''

\textbf{Pugnacious}: eager or quick to argue, quarrel, or fight. ``the increasingly pugnacious demeanor of politicians.''

\textbf{Heyday}: the period of a person's or thing's greatest success, popularity, or vigor. ``the paper has lost millions of readers since its heyday in 1964.''

\textbf{Amok}: behave uncontrollably and disruptively. ``stone-throwing anarchists running amok.''

\textbf{Badger}: ask (someone) repeatedly and annoyingly for something; pester. ``journalists badgered him about the deals.''

\textbf{Rile}: make (someone) annoyed or irritated. ``it was his air of knowing all the answers that riled her.''

\textbf{Peevish}: easily irritated, especially by unimportant things. ``all this makes Steve fretful and peevish.''

\textbf{Fractious}: irritable and is likely to cause disruption. ``We rarely invite my fractious Uncle over for dinner; he always complains about the food, and usually launches into a tirade on some touchy subject.''

\textbf{Bilious}: spiteful; bad-tempered. ``outbursts of bilious misogyny.''

\textbf{Petulant}: (of a person or their manner) childishly sulky or bad-tempered. ``he was moody and petulant.''

\textbf{Mordant}: (especially of humor) having or showing a sharp or critical quality; biting. ``a mordant sense of humor.''

\textbf{Fretful}: feeling or expressing distress or irritation. ``the baby was crying with a fretful whimper.''

\textbf{Exhort}: To strongly urge on; encourage. ``the media have been exhorting people to turn out for the demonstration.'' - ``Erdo\u{g}an exhorted people to protest during the coup.''

\textbf{Mettlesome}: (of a person or animal) full of spirit and courage. ``For its raid on the Bin Laden’s compound in Abbottabad, Seal Team Six has become, for many Americans, the embodiment of mettle.''

\textbf{Nettlesome}: causing annoyance or difficulty. ``complicated and nettlesome regional disputes.''

\textbf{Corroborate}: To confirm or lend support to (usually an idea or claim). ``Her claim that frog populations were falling precipitously in Central America was corroborated by locals, who reported that many species of frogs had seemingly vanished overnight.''

\textbf{Cupidity}: greed for money or possessions. ``In reality, the prospect is implausible: reduce a man's propensity to lust and he will compensate with an increased aggression or cupidity .''

\textbf{Avarice}: Greed. ``Being free from avarice, the material wealth has absolutely no significance for Shiva.''

\textbf{Relegate}: consign or dismiss to an inferior rank or position. ``they aim to prevent women from being relegated to a secondary role.''

\textbf{Devolve}: transfer or delegate (power) to a lower level, especially from central government to local or regional administration; grow worse. ``measures to devolve power to the provinces.'' - ``The dialogue between the two academics devolved into a downright bitter argument.''

\textbf{Arriviste}: an ambitious or ruthlessly self-seeking person, especially one who has recently acquired wealth or social status. ``In Thackeray's next full-length novel, the Newcomes are so called because they are both a nouveau riche and an arriviste family.''

\textbf{Lassitude}: a state of physical or mental weariness; lack of energy. ``she was overcome by lassitude and retired to bed.''

\textbf{Travail}: painful or laborious effort. ``advice for those who wish to save great sorrow and travail.''

\textbf{Scintillating}: sparkling or shining brightly, brilliant. ``the scintillating sun.''

\textbf{Shimmer}: a soft, slightly wavering light. ``a pale shimmer of moonlight.''

\textbf{Steadfast}: resolutely or dutifully firm and unwavering. ``steadfast loyalty.''

\textbf{Incessant}: (of something regarded as unpleasant) continuing without pause or interruption. ``the incessant beat of the music.''

\textbf{Ingenuity}: the quality of being clever, original, and inventive. ``In every way - performance, build quality, ingenuity of design - it is the better car.''

\textbf{Vacuous}: having or showing a lack of thought or intelligence; mindless. (anlamsiz). ``A vacuous smile.''

\textbf{Artlessness}: the quality of innocence. ``I, personally, found the artlessness of her speech charming.''

\textbf{Ingenuousness}: innocent, artless, simple. ``Ingenuous person, that is I.''

\textbf{Proponent}: a person who advocates a theory, proposal, or project. ``a collection of essays by both critics and proponents of graphology.''

\textbf{Advocate}: a person who publicly supports or recommends a particular cause or policy; publicly recommend or support. ``he was an untiring advocate of economic reform.'' - ``they advocated an ethical foreign policy.''

\textbf{Plodding}: slow-moving and unexciting; walk doggedly and slowly with heavy steps. ``a plodding comedy drama.'' - ``we plodded back up the hill.''

\textbf{Ponderous}: slow and clumsy because of great weight. ``her footsteps were heavy and ponderous''

\textbf{Pith}: the essence of something. ``a book that he considered contained the pith of all his work.''

\textbf{Pithy}: (of language or style) concise and forcefully expressive. ``Fighting for the Future, for all its provocative arguments and pithy language, sometimes borders on the apocalyptic.''

\textbf{Obtain}: be valid, applicable, or true. ``The custom of waiting your turn in line does not obtain in some countries, in which many people try to rush to front of the line at the same time.''

\textbf{Veritable}: used as an intensifier, often to qualify a metaphor; truthfully, without a doubt. ``Frank is a veritable life-saver -- last year, on two different occasions, he revived people using CPR.''

\textbf{Maxim}: a short, pithy statement expressing a general truth or rule of conduct; apothegm; aphorism. ``the maxim that actions speak louder than words.''

\textbf{Aphoristic}: of, like, or containing aphorisms; something that is concise and instructive of a general truth or principle. ``Sometimes I can't stand Nathan because he tries to impress everyone by being aphoristic, but he just states the obvious.''

\textbf{Erudite}: Having or showing profound knowledge. ``Before the Internet, the library was typically where you would find erudite readers.''

\textbf{Astute}: having or showing an ability to accurately assess situations or people and turn this to one's advantage. ``Alfonzo was an astute chess player, always ready to use an opponent's misplay as an opportunity to attack.''

\textbf{Consummate}: showing a high degree of skill and flair; complete or perfect.  ``she dressed with consummate elegance.''

\textbf{Savvy}: shrewdness and practical knowledge, especially in politics or business. ``the financiers lacked the necessary political savvy.''

\textbf{Impeccable}: Without fault or error. ``He was impeccably dressed in the latest fashion without a single crease or stain.''

\textbf{Paragon}: model of excellence or perfection of a kind; one having no equal; an ideal instance; a perfect embodiment of a concept. ``Even with the rise of Kobe Bryant, many still believe that Michael Jordon is the paragon for basketball players.''

\textbf{Stolid}: (of a person) calm, dependable, and showing little emotion or animation. ``Too much anti-depressants make people stolid.''

\textbf{Raffish}: unconventional and slightly disreputable, especially in an attractive manner. ``His raffish air.''

\textbf{Grandiloquent}: pompous or extravagant in language, style, or manner, especially in a way that is intended to impress. ``The dictator was known for his grandiloquent speeches, puffing his chest out and using big, important-sounding words.''

\textbf{Muted}: Softened, subdued. ``Helen preferred muted earth colors, such as green and brown, to the bright pinks and red her sister liked.''

\textbf{Subdued}: (of a person or their manner) quiet and rather reflective or depressed. ``I felt strangely subdued as I drove home.''

\textbf{Avid}: Marked by active interest and enthusiasm. ``An avid gamer.''

\textbf{Boon}: a thing that is helpful or beneficial;  (of a companion or friend) close, intimate, favorite. ``may I have the inestimable boon of a few minutes' conversation?'' - ``he debated the question with a few boon companions in the bar room.''

\textbf{Estimable}: deserving of esteem and respect. ``After serving thirty years, in which he selflessly served the community, Judge Harper was one of the more estimable people in town.''

\textbf{Reservation}: an unstated doubt that prevents you from accepting something wholeheartedly. ``I was initially excited by the idea of a trip to Washington, D.C. but now that I have read about the high crime statistics there, I have some reservations.''

\textbf{Perturb}: make (someone) anxious or unsettled. ``they were perturbed by her capricious behavior.''

\textbf{Irk}: irritate; annoy. ``it irks her to think of the runaround she received.''

\textbf{Exasperate}: irritate intensely; infuriate. ``As a child, I exasperated my mother with strings of never-ending questions.''

\textbf{Vex}: make (someone) feel annoyed, frustrated, or worried, especially with trivial matters. ``the memory of the conversation still vexed him.''

\textbf{Vie}: compete eagerly with someone in order to do or achieve something. ``rival mobs vying for control of the liquor business.''

\textbf{Eke}: manage to support oneself or make a living with difficulty. ``they eked out their livelihoods from the soil.''

\textbf{Elude}: evade or escape from (a danger, enemy, or pursuer), typically in a skillful or cunning way. ``he managed to elude his pursuers by escaping into an alley.''

\textbf{Balk}: hesitate or be unwilling to accept an idea or undertaking. ``any gardener will at first balk at enclosing the garden.''

\textbf{Coax}: persuade (someone) gradually or by flattery to do something. ``the trainees were coaxed into doing hard, boring work.''

\textbf{Disabuse}: persuade (someone) that an idea or belief is mistaken. ``he quickly disabused me of my fanciful notions.''

\textbf{Proselytize}: convert or attempt to convert (someone) from one religion, belief, or opinion to another. ``the program did have a tremendous evangelical effect, proselytizing many.'' 

\textbf{Obstinate}: stubbornly refusing to change one's opinion or chosen course of action, despite attempts to persuade one to do so. ``You can have really strong, obstinate opinions, so long as your facts are true, you're OK.''

\textbf{Benighted}: in a state of pitiful or contemptible intellectual or moral ignorance, typically owing to a lack of opportunity. ``they saw themselves as bringers of culture to poor benighted peoples.''

\textbf{Blinkered}: to have a limited outlook or understanding. ``In gambling, the blinkered addict is easily influenced by past successes and/or past failures, forgetting that the outcome of any one game is independent of the games that preceded it.''

\textbf{Obtuse}: annoyingly insensitive or slow to understand; an angle greater than $90^\circ$ and less than $180^\circ$. (genis aci) ``he wondered if the doctor was being deliberately obtuse.''

\textbf{Enmity}: the state or feeling of being actively opposed or hostile to someone or something. ``At least, we don't feel enmity toward fellow human beings very often.''

\textbf{Incongruous}: not in harmony or keeping with the surroundings or other aspects of something. ``the duffel coat looked incongruous with the black dress she wore underneath.''

\textbf{Discord}: disagreement between people; lack of agreement or harmony. ``Despite all their talented players, the team was filled with discord--some players refused to talk to others--and lost most of their games.''

\textbf{Collusion}: secret or illegal cooperation or conspiracy, especially in order to cheat or deceive others. ``the armed forces were working in collusion with drug traffickers.''

\textbf{Ploy}: a cunning plan or action designed to turn a situation to one's own advantage.  ``the president has dismissed the referendum as a ploy to buy time.''

\textbf{Finagle}: obtain (something) by devious or dishonest means. ``Ted attended all the football games he could finagle tickets for.''

\textbf{Proscribe}: forbid, especially by law. ``My doctor proscribed my habit of eating donuts with chocolate sauce and hamburger patties for breakfast.''

\textbf{Ascribe}: attribute something to (a cause). ``he ascribed Jane's short temper to her upset stomach.''

\textbf{Impute}: represent (something, especially something undesirable) as being done, caused, or possessed by someone; attribute. ``He imputed his subpar performance on the test to a combination of stress and poor sleep.''

\textbf{Impugn}: attack as false or wrong. ``Though many initially tried to impugn Darwin's theory, in scientific circles today, the idea is taken as truth.''

\textbf{Docile}: ready to accept control or instruction; submissive. ``They turned to run, but the creatures seemed docile and did not attack.''

\textbf{Rakish}: having or displaying a dashing, jaunty, or slightly disreputable quality or appearance. ``he had a rakish, debonair look.''

\textbf{Debonair}: (of a man) confident, stylish, and charming. ``He's debonair, smooth, handsome and slim like Moore.''

\textbf{Incumbent}: necessary for (someone) as a duty or responsibility. ``it is incumbent on all decent people to concentrate on destroying this evil.''

\textbf{Behoove}: it is a duty or responsibility for someone to do something; it is incumbent on. ``it behooves any coach to study his predecessors.''

\textbf{Desecrate}: treat (a sacred place or thing) with violent disrespect; violate. ``more than 300 graves were desecrated.''

\textbf{Consecrate}: make or declare (something, typically a church) sacred; dedicate formally to a religious or divine purpose. ``the present Holy Trinity church was consecrated in 1845.''

\textbf{Analogous}: comparable in certain respects, typically in a way that makes clearer the nature of the things compared. ``they saw the relationship between a ruler and his subjects as analogous to that of father and children.''

\textbf{Uncanny}: strange or mysterious, especially in an unsettling way. ``an uncanny feeling that she was being watched.''

\textbf{Begrudge}: envy (someone) the possession or enjoyment of (something); give reluctantly or resentfully. ``she begrudged Martin his affluence.'' - ``nobody begrudges a single penny spent on health.''

\textbf{Chary}: cautiously or suspiciously reluctant to do something. ``most people are chary of allowing themselves to be photographed.''

\textbf{Coalesce}: come together and form one mass or whole. ``Over time, the various tribes coalesced into a single common culture with one universal language.''

\textbf{Convene}: come or bring together for a meeting or activity; assemble. ``he convened a group of well-known scientists and philosophers.''

\textbf{Cohesive}: Well integrated, forming a united whole. ``A well-written, cohesive essay will keep on topic at all times, never losing sight of the main argument.''

\textbf{Coherent}: (of an argument, theory, or policy) logical and consistent. ``they failed to develop a coherent economic strategy.''

\textbf{Prevail}: be widespread in a particular area at a particular time, be current; prove superior. ``During the labor negotiations, an air of hostility prevailed in the office.'' - ``Before the cricket match, Australia was heavily favored, but India prevailed.''

\textbf{Presumptuous}: (of a person or their behavior) failing to observe the limits of what is permitted or appropriate. ``I hope I won't be considered presumptuous if I offer some advice.''

\textbf{Antedate}: precede in time; come before (something) in date. ``a civilization that antedated the Roman Empire.''

\textbf{Champion}: to fight for a cause. ``Martin Luther King Jr. championed civil rights fiercely throughout his short life.''

\textbf{Celerity}: swiftness of movement. ``An individual's crime calculus is influenced by three factors: certainty, severity, and celerity of punishment.''

\textbf{Dispatch}: deal with (a task, problem, or opponent) quickly and efficiently. ``they dispatched the opposition.''

\textbf{Resignation}: the acceptance of something undesirable but inevitable. ``A shrug of resignation.''

\textbf{Amply}: More than is adequate (fazlasiyla). ``The bot was amply supplied, no man would go hungry or thirsty.''

\textbf{Deliberate}: done consciously and intentionally; engage in long and careful consideration. ``a deliberate attempt to provoke conflict.'' - ``she deliberated over the menu.''

\textbf{Sedulous}: (of a person or action) showing dedication and diligence. ``An avid numismatist, Harold sedulously amassed a collection of coins from over 100 countries—an endeavor that took over fifteen years across five continents.''

\textbf{Nadir}: The lowest point. ``For many pop music fans, the rap– and alternative-rock–dominated 90s were the nadir of musical expression.''

\textbf{Zeitgeist}: the spirit of the times. ``the story captured the zeitgeist of the late 1960s.''

\textbf{Anachronism}: something that is inappropriate for the given time period (usually something old). ``Dressed in 15'th century clothing each day, Edward was a walking anachronism.''

\textbf{Apogee}: Highest point, zenith, apex, summit, pinnacle, climax. ``The apogee of the Viennese style of music, Mozart’s music continues to mesmerize audiences well into the 21st century.''

\textbf{Turpitude}: depravity; wickedness. ``The moral turpitude of youth is, and always has been, offensive to its elders.''

\textbf{Arrant}: complete, utter. ``Did you ever, in all your life, hear such arrant nonsense?''

\textbf{Exegesis}: critical explanation or analysis, especially of a text (or a holy book). ``The Bible is fertile ground for exegesis.''

\textbf{Expunge}: to eliminate completely. ``When I turned 18, all of the shoplifting and jaywalking charges were expunged from my criminal record.''

\textbf{Byzantine}: (of a system or situation) excessively complicated, typically involving a great deal of administrative detail. ``Byzantine insurance regulations.''

\textbf{Cow}: to intimidate. ``Do not be cowed by a 3000 word vocabulary!''

\textbf{Imbroglio}: an extremely confused, complicated, or embarrassing situation. ``The corruption imbroglio may be one scandal too far for the Tax Commissioner.''

\textbf{Sententious}: given to moralizing in a pompous or affected manner. ``he tried to encourage his men with sententious rhetoric.''

\textbf{Quisling}: a traitor who collaborates with an enemy force occupying their country. ``In Turkey it is really easy to be labelled a quisling.''

\textbf{Gerrymander}: manipulate the boundaries of (an electoral constituency) so as to favor one party or class. ``Umno continues to benefit from a gerrymander that favours rural Malay seats on peninsular Malaya as well as Sabah and Sarawak in northern Borneo.''

\textbf{Schadenfreude}:  joy from watching the suffering and misfortune of others. ``From his warm apartment window, Stanley reveled in schadenfreude as he laughed at the figures below, huddled together in the arctic chill.''

\textbf{Jaundiced}: affected by bitterness, resentment, or cynicism. ``they looked on politicians with a jaundiced eye.''

\textbf{Pyrrhic}: (of a victory) won at too great a cost to have been worthwhile for the victor. ``Unless that is done, any military success in Afghanistan will be a pyrrhic victory.''

\textbf{Malapropism}: the confusion of a word with another word that sounds similar. ``Whenever I looked glum, my mother would offer to share "an amusing antidote" with me—an endearing malapropism of "anecdote" that never failed to cheer me up.''

\textbf{Palimpsest}: something that has been changed numerous times but on which traces of former iterations can still be seen. ``The downtown was a palimpsest of the city’s checkered past.''

\textbf{Quixotic}: exceedingly idealistic; unrealistic and impractical. ``a vast and perhaps quixotic project.''

\textbf{Venial}: easily excused or forgiven; pardonable. ``His traffic violations ran the gamut from the venial to the egregious.''

\textbf{Factitious}: artificially created or developed. ``The defendant’s story was largely factitious and did not accord with eyewitness testimonies.''

\textbf{Fell}: terribly evil. ``sorcerers use spells to achieve their fell ends.''

\textbf{Tendentious}: expressing or intending to promote a particular cause or point of view, especially a controversial one. ``They make some good points, some misleading points, and a few rather tendentious points.''

\textbf{Arch}: deliberately or affectedly playful and teasing. ``arch observations about even the most mundane matters.''

\textbf{Parvenu}: a person of obscure origin who has gained wealth, influence, or celebrity. ``The theater was full of parvenus who each thought that they were surrounded by true aristocrats.''

\textbf{Prolixity}: boring verbosity, using too much words. ``I loved my grandfather dearly, but his prolixity would put me to sleep, regardless of the topic.''

\textbf{Laconic}: (of a person, speech, or style of writing) using very few words. ``his laconic reply suggested a lack of interest in the topic.''

\textbf{Illustrious}: well known, respected, and admired for past achievements. ``his illustrious predecessor.''

\textbf{Semblance}: an outward or token appearance or form that is deliberately misleading. ``While the banker maintained a semblance of respectability in public, those who knew him well were familiar with his many crimes.''

\textbf{Gumption}: resourcefulness and determination. ``Wallace Stegner lamented the lack of gumption in the U.S. during the sixties, claiming that no young person knew the value of work.''

\textbf{Plucky}: having or showing determined courage in the face of difficulties. ``But seconds later the plucky rider had regained her composure and remounted her animal.''

\textbf{Enthrall}: capture the fascinated attention of. ``she had been so enthralled by the adventure that she had hardly noticed the cold.''

\textbf{Probity}: integrity, strong moral principles. ``The ideal politician would have the probity to lead, but reality gravely falls short of the ideal of morally upright leaders.''

\textbf{Sinecure}: a position requiring little or no work but giving the holder status or financial benefit, easy job. ``One insider linked with the private security business said: ‘All these jobs are a nice sinecure for a cop.''

\textbf{Litany}: any long and tedious account of something. ``Mr. Rogers spoke to a Senate committee and did not give a litany of reasons to keep funding the program, but instead, appealed to the basic human decency of all present.''

\textbf{Primacy}: the fact of being primary, preeminent, or more important, the state of being first in importance. ``The primacy of Apple Computers is not guaranteed, as seen in the recent lawsuits and weak growth.''

\textbf{Enjoin}: instruct or urge (someone) to do something / prohibit. ``the code enjoined members to trade fairly.'' - ``They enjoined chemical warfare.''

\textbf{Moribund}: (of a person) at the point of death. ``Whether you like it or not, jazz as a genre is moribund at best, possibly already dead.''

\textbf{Insouciance}: casual lack of concern; indifference. ``an impression of boyish insouciance.''

\textbf{Flummox}: perplex (someone) greatly; bewilder. ``he was completely flummoxed by the question.''

\textbf{Raconteur}: a person who tells anecdotes in a skillful and amusing way. ``Nasreddin Hoca is a raconteur.''

\textbf{Portentous}: of or like a portent; of momentous significance. ``this portentous year in Canadian history.''

\textbf{Portent}: a sign or warning that something, especially something momentous or calamitous, is likely to happen. / an exceptional or wonderful person or thing. ``they believed that wild birds in the house were portents of death.'' - ``what portent can be greater than a pious notary?''

\textbf{Redoubtable}: (of a person) formidable, especially as an opponent, inspiring fear or awe. ``he was a redoubtable debater.''

\textbf{Charlatan}: a person falsely claiming to have a special knowledge or skill; a fraud. ``Take the case of the charlatan who claims to transmit thoughts at a distance.''

\textbf{Cede}: give up (power or territory). ``they have had to cede control of the schools to the government.''

\textbf{Bowdlerize}: remove material that is considered improper or offensive from (a text or account), especially with the result that it becomes weaker or less effective. ``a bowdlerized version of the story.''

\textbf{Ossify}: cease developing; be stagnant or rigid. ``Even as a young man, Bob had some bias against poor people, but during his years in social services, his bad opinions ossified into unshiftable views.''

\textbf{Peripatetic}: traveling from place to place, especially working or based in various places for relatively short periods. ``Perhaps the royal colleges should appoint peripatetic experts who would travel around the country.''

\textbf{Untenable}: (especially of a position or view) not able to be maintained or defended against attack or objection. ``With the combination of Kepler's brilliant theories and Galileo's telescopic observations, the old geocentric theory became untenable to most of the educated people in Europe.''

\textbf{Refractory}: stubborn or unmanageable. resistant to a process or stimulus. ``some granules are refractory to secretory stimuli.''

\textbf{Obstreperous}: noisily and stubbornly defiant; willfully difficult to control. ``When the teacher asked the obstreperous student simply to bus his tray, the student threw the entire tray on the floor, shouted an epithet, and walked out.''

\textbf{Stalwart}: loyal, reliable, and hardworking. ``he remained a stalwart supporter of the cause.''

\textbf{Puissant}: having great power or influence. ``Faith is a puissant concept, yet it is problematic in isolation.''

\textbf{Dovetail}: fit together tightly, as if by means of an interlocking joint. ``Although Darwin's evolution and Mendel's genetics were developed in isolation from one another, they dovetail very well.''

\textbf{Presentiment}: an intuitive feeling about the future, especially one of foreboding: fearful apprehension; a feeling that something bad will happen. ``On the night that Lincoln would be fatally shot, his wife had a presentiment about going to Ford's Theater, but Lincoln persuaded her that everything would be fine.''

\textbf{Machinate}: engage in plots and intrigues; scheme. ``The rebels met at night in an abandoned barn to machinate.''

\textbf{Inviolate}: free or safe from injury or violation. ``an international memorial which must remain inviolate.''

\textbf{Incontrovertible}: not able to be denied or disputed. ``incontrovertible proof.''

\textbf{Puerile}: childishly silly and trivial, immature. ``you're making puerile excuses.''

\textbf{Magisterial}: having or showing great authority. ``a magisterial pronouncement.''

\textbf{Spartan}: unsparing and uncompromising in discipline or judgment; practicing great self-denial. ``After losing everything in a fire, Tim decided to live in spartan conditions, sleeping on the floor and owning as little furniture as a possible.''

\textbf{Invidious}: (of an action or situation) likely to arouse or incur resentment or anger in others. ``she'd put herself in an invidious position.''

\textbf{Meteoric}: like a meteor in speed or brilliance or transience. ``The early spectacular successes propelled the pitcher to meteoric stardom, but a terrible injury tragically cut short his career.''

\textbf{Untrammeled}: not deprived of freedom of action or expression; not restricted or hampered. ``a mind untrammelled by convention.''

\textbf{Hoary}: grayish white / old and trite. ``hoary cobwebs.'' - ``that hoary American notion that bigger is better.''

\textbf{Inanity}: a nonsensical remark or action, silliness. ``they utter whatever inanities will get them elected.''

\textbf{Ersatz}: (of a product) made or used as a substitute, typically an inferior one, for something else. ``ersatz coffee.''

\textbf{Chimera}: something desired or wished for but is only an illusion and impossible to achieve. ``Many believe that a world free of war is a chimera—a dream that ignores humanity's violent tendencies.''

\textbf{Disingenuous}: not straightforward; giving a false appearance of frankness. ``Many adults think that they can lie to children, but kids are smart and know when people are disingenuous.''

\textbf{Deign}: do something that one considers to be beneath one's dignity. ``she did not deign to answer the maid's question.''

\textbf{Brook}: put up with something or somebody unpleasant. ``While she was at the chalkboard, the teacher did not brook any form of talking--even a tiny peep resulted in afternoon detention.''

\textbf{Epigram}: A witty saying. ``My favorite epigram from Mark Twain is "A man who carries a cat by the tail learns something he can learn no other way.''

\textbf{Bristle}: be covered with or abundant in. ``the roof bristled with antennas.''

\textbf{Unprepossessing}: creating an unfavorable or neutral first impression, not particularly attractive or appealing to the eye. ``World leaders coming to meet Gandhi would expect a towering sage, and often would be surprised by the unprepossessing little man dressed only in a loincloth and shawl.''

\textbf{Baleful}: threatening or foreshadowing evil or tragic developments. ``Movies often use storms or rain clouds as a baleful omen of evil events that will soon befall the main character.''

\textbf{Appurtenant}: belonging; pertinent; supplying added support. ``In hiking Mt. Everest, the Sherpa are appurtenant, helping climbers both carry gear and navigate treacherous paths.''

\textbf{Languish}: (of a person or other living thing) lose or lack vitality; grow weak or feeble; suffer from being forced to remain in an unpleasant place or situation. ``he has been languishing in jail since 1974.''

\textbf{Effervescent}: vivacious and enthusiastic. ``effervescent young people.''

\textbf{Atavism}: a reappearance of an earlier characteristic; throwback. ``Much of the modern art movement was an atavism to a style of art found only in small villages through Africa and South America.''

\textbf{Inchoate}: only partly in existence; imperfectly formed. ``just begun and so not fully formed or developed; rudimentary.''

\textbf{Ineffable}: too great or extreme to be expressed or described in words. ``While art critics can occasionally pinpoint a work's greatness, much of why a piece captures our imaginations is completely ineffable.''

\textbf{Arrogate}: seize and control without authority. ``Arriving at the small town, the outlaw arrogated the privileges of a lord, asking the frightened citizens to provide food, drink, and entertainment.''

\textbf{Squelch}: suppress or crush completely. ``After the dictator consolidated his power, he took steps to squelch all criticism, often arresting any journalist who said anything that could be interpreted as negative about his regime.''

\textbf{Assiduous}: showing great care and perseverance. ``she was assiduous in pointing out every feature.''

\textbf{Pontificate}: express one's opinions in a way considered annoyingly pompous and dogmatic. ``he was pontificating about art and history.''

\textbf{Complicit}: Associated with or participating in an activity, especially one of a questionable nature. ``While the grand jury cleared the senator of all criminal charges, in the public mind he was still complicit in the corruption.''

\textbf{Panegyric}: a public speech or published text in praise of someone or something. ``Vera's panegyric on friendship.''

\textbf{Impervious}: not admitting of passage or capable of being affected. ``I am not impervious to your insults; they cause me great pain.''

\textbf{Bridle}: the act of restraining power or action or limiting excess / to react with anger or to take offense. ``New curfew laws have bridled people's tendency to go out at night.'' - ``The hostess bridled at the tactless dinner guests who insisted on eating before everybody had gotten their food.''

\textbf{Abjure}: solemnly renounce (a belief, cause, or claim). (tovbe etmek). ``While the church believed that Galileo abjured the heliocentric theory under threat of torture, he later wrote a book clearly supporting the theory.''

\textbf{Complaisant}: willing to please others; obliging; agreeable. ``when unharnessed, Northern dogs are peaceful and complaisant.''

\textbf{Pellucid}: translucently clear. ``The professor had a remarkable ability to make even the most difficult concepts seem pellucid.''

\textbf{Inviolable}: never to be broken, infringed, or dishonored. ``an inviolable rule of chastity.''

\textbf{Desideratum}: something that is needed or wanted. ``The desideratum of the environmental group is that motorists should rely on carpooling.''

\textbf{Inequity}: lack of fairness or justice. ``policies aimed at redressing racial inequity.''

\textbf{Enormity}: the great or extreme scale, seriousness, or extent of something perceived as bad or morally wrong. ``a thorough search disclosed the full enormity of the crime.''

\textbf{Coterminous}: having the same boundaries or extent in space, time, or meaning. ``the southern frontier was coterminous with the French Congo colony.''

\textbf{Concomitant}: describing an event or situation that happens at the same time as or in connection with another; naturally accompanying or associated. (eslik eden). ``Concomitant with his desire for nature was a desire for the culture and energy of a big city.''

\textbf{Flag}: (of a person) become tired, weaker, or less enthusiastic. ``if you begin to flag, there is an excellent cafe to revive you.''

\textbf{Stultify}: cause to lose enthusiasm and initiative, especially as a result of a tedious or restrictive routine. ``the mentally stultifying effects of a disadvantaged home.''

\textbf{Malinger}: exaggerate or feign illness in order to escape duty or work. ``Earlier this year a poll found that 40\% of small businesses thought employees were malingering when they took sick leave.''

\textbf{Ethereal}: characterized by lightness and insubstantiality. ``Because she dances with an ethereal style, ballet critics have called her Madame Butterfly.''

\textbf{Aroma}: an enjoyable smell. ``A beautiful aroma.''

\textbf{Fragrant}: having a strong, good smell. ``Shirley loved growing roses because they were much more fragrant than the other, more dull smelling flowers in her garden.''

\textbf{Noisome}: having an extremely offensive smell. ``noisome vapors from the smoldering waste.''

\textbf{Self-effacing}: reluctant to draw attention to yourself; not claiming attention for oneself; retiring and modest. ``The most admirable teachers and respected leaders are those who are self-effacing, directing attention and praise to their students and workers.''

\textbf{Desiccated}: lacking interest, passion, or energy. ``Few novelists over 80 are able to produce anything more than desiccated works--boring shadows of former books.''

\textbf{Conflagration}: a very intense and uncontrolled fire. ``In the summer months, conflagrations are not uncommon in the southwest, due to the heat and lack of rain.''

\textbf{Palatable}: acceptable to the taste or mind. ``Mikey didn't partake much in his friends' conversations, but found their presence palatable.''

\textbf{Internecine}: (of conflict) within a group or organization, destructive to both sides in a conflict. ``The guerilla group, which had become so powerful as to own the state police, was finally destroyed by an internecine conflict.''

\textbf{Canard}: an unfounded rumor or story. ``the old canard that LA is a cultural wasteland.''

\textbf{Splenetic}: bad-tempered; spiteful; very irritable. ``Ever since the car accident, Frank has been unable to walk without a cane, and so he has become splenetic and unpleasant to be around.''

\textbf{Extrapolate}: draw from specific cases for more general cases. ``By extrapolating from the data on the past three months, we can predict a 5\% increase in traffic to our website.''

\textbf{Verisimilitude}: the appearance of being true or real. ``All bad novels are bad for numerous reasons; all good novels are good for their verisimilitude of reality, placing the readers in a world that resembles the one they know.''

\textbf{Row}: a noisy acrimonious quarrel, an angry dispute. ``they had a row and she stormed out of the house.''

\textbf{Palaver}: speak (about unimportant matters) rapidly and incessantly; talk unnecessarily at length. (palavra). ``During the rain delay, many who had come to see the game palavered, probably hoping that idle chatter would make the time go by faster.''

\textbf{Promulgate}: promote or make widely known (an idea or cause). ``these objectives have to be promulgated within the organization.''

\textbf{Officious}: intrusive in a meddling or offensive manner. ``The professor had trouble concentrating on her new theorem, because her officious secretary would barge in frequently reminding her of some trivial detail involving departmental paperwork.''

\textbf{Ebullient}: joyously unrestrained, cheerful and full of energy. ``Can you blame him for his ebullient mood? He just graduated from medical school.''

\textbf{Capitulate}: cease to resist an opponent or an unwelcome demand; surrender (usually under agreed conditions). ``Paul, losing 19-0 in a ping-pong match against his nimble friend, basically capitulated when he played the last two points with his eyes closed.''

\textbf{Eponym}: a person after whom a discovery, invention, place, etc., is named or thought to be named. ``Alexandria, Egypt is an eponym because it is named after Alexander the Great.''

\textbf{Adjudicate}: make a formal judgment or decision about a problem or disputed matter. ``we asked him to adjudicate at the local flower show.''

\textbf{Restive}: restless. ``The crowd grew restive as the comedians opening jokes fell flat.''

\textbf{Subsume}: include or absorb (something) in something else. ``most of these phenomena can be subsumed under two broad categories.''

\textbf{Galvanize}: shock or excite (someone), typically into taking action. ``My love galvanized me during the marathon.''

\textbf{Parochial}: narrow-minded. ``His tastes are too parochial.''

\textbf{Prevaricate}: Speak evasively. ``The cynic quipped, ``There is not much variance in politicians, they all seem to prevaricate.'''' - ``Ismet pasha vs Cicerin, cicerin was prevaricating.''

\textbf{Tint}: a shade or variety of color. (renk tonu). ``the sky was taking on an apricot tint.''

\textbf{Gaudy}: extravagantly bright or showy, typically so as to be tasteless. ``Karen knew the bright pink, skin-tight suit was gaudy, but she had promised her five-year old son that he could pick her clothing for the party.''

\textbf{Endeavour}: try hard to achieve something, an attempt to achieve something. ``an endeavor to reduce serious injury.'' 

\textbf{Institute}: to start (a rule or system). ``In January, we will institute a plan to make car accidents less frequent.''

\textbf{Constitute}: be (a part) of a whole. ``single parents constitute a great proportion of the poor.''

\textbf{Facilitate}: make easier. ``schools were located on the same campus to facilitate the sharing of resources.''

\textbf{Instrumental}: acting as a key part of a process. ``The university believed that passion was instrumental to learning—without it, you'd be wasting your time in class.''

\textbf{Incentive}: a thing that motivates or encourages one to do something. ``there is no incentive for customers to conserve water.''

\textbf{Preposterous}: contrary to reason or common sense; utterly absurd or ridiculous. ``a preposterous suggestion.''

\textbf{Grave}: very serious. ``Most of Bach's orchestral melodies are light, but his organ music often has a grave quality that fits movie scenes of suspense or horror.''

\textbf{Encroach}: intrude on (a person's territory or a thing considered to be a right). ``The US President's ability to give military orders without declaring war encroaches on the Congress's authority in giving the formal declaration.''

\textbf{Entice}: to attract or persuade with promises of benefits. ``Some of the most effective hunters are those who entice their targets into traps with the appearance or smell of food.''

\textbf{Recede}: to go back to a previous position. ``Before a tsunami hits, the water recedes, which is dangerous for those who unknowingly go to investigate before the big waves come rushing in.''

\textbf{Waive}: refrain from insisting on or using (a right or claim). ``he will waive all rights to the money.''

\textbf{Intersperse}: scatter among or between other things; place here and there. ``interspersed between tragic stories are a few songs supplying comic relief.''

\textbf{Outlook}: a person's point of view or general attitude to life. ``broaden your outlook on life.''

\textbf{Luddite}: a person opposed to new technology or ways of working. ``There is no need to retreat to a Luddite attitude to new things, but rather embrace a hopeful posture to the possibilities that technology provides for new avenues of human imagination.''

\textbf{Amend}: to make small changes (to a text) to improve. ``The lawyer suggested amending the current contract rather than starting work on an entirely new deal.''

\textbf{Calamity}: an event causing great and often sudden damage or distress; a disaster. ``the fire was the latest calamity to strike the area.''

\textbf{Perpetuate}: to cause to continue. ``If you do not let him do things for himself, you are merely perpetuating bad habits that will be even harder to break in the future.''

\textbf{Confer}: to meet and discuss before deciding. ``Before I invite you to join our secret club, I'll have to confer with the other members and get their approval.''

\textbf{Procure}: obtain (something), especially with care or effort. ``food procured for the rebels.''

\textbf{Predominant}: present as the strongest or main element. ``its predominant color was white.''

\textbf{Oblong}: having an elongated shape, as a rectangle or an oval. ``A table of oblong shape stood midway between the drawing-room walls.''

\textbf{Pseudonym}: a fictitious name, especially one used by an author. ``I wrote under the pseudonym of Evelyn Hervey.''

\textbf{Inept}: having or showing no skill; clumsy. ``the inept handling of the threat.''

\textbf{Beckon}: to invite or attract to come to. ``Julie was supposed to spend all day in the library, but the sun beckoned as she passed a window and she couldn't resist going outside.''

\textbf{Brittle}: hard but liable to break or shatter easily. ``her bones became fragile and brittle.''

\textbf{Forgo}: omit or decline to take (something pleasant or valuable); go without. ``she wanted to forgo the dessert and leave while they could.''

\textbf{Advent}: the arrival of a notable person, thing, or event. ``the advent of television.''

\textbf{Wither}: (of a plant) become dry and shriveled / cause harm or damage to. ``the grass had withered to an unappealing brown.'' - ``a business that can wither the hardiest ego.''

\textbf{Torrent}: An extremely heavy rain. ``The torrent of rain that accompanied the typhoon dumped several metres of water on the city over the course of the week.''

\textbf{Hoist}: an act of raising or lifting something. ``In the United States, it is stipulated by law that public institutions such as schools will hoist the national flag.''

\textbf{Pressing}: important immediately. ``It was six in the morning and she still hadn't finished the essay that was due at eight; but after working all night, her most pressing need was a cup of coffee.''

\textbf{Levy}: an act of levying a tax, fee, or fine. (haciz). ``union members were hit with a 2 percent levy on all pay.''

\textbf{Cull}: select from a large quantity; obtain from a variety of sources; to remove something unwanted from a group. ``Because we had 67 applications for the 20 positions on the committee, we began by culling all who clearly lacked even the minimum experience.''

\textbf{Trappings}: the visual objects and signs associated with a position or status. ``I had the trappings of success.''

\textbf{Raze}: completely destroy (a building, town, or other site). ``villages were razed to the ground.''

\textbf{Adhere}: stick fast to (a surface or substance). ``paint won't adhere well to a greasy surface.''

\textbf{Spearhead}: lead (an attack or movement or change). ``After volunteering at a homeless shelter, I decided to spearhead a volunteer program at my office to get more of my coworkers involved.''

\textbf{Trickle}: a small flow of liquid. ``I watched the raindrop trickle down the car window as we sat stopped at a stoplight.''

\textbf{Endow}: give or bequeath an income or property to (a person or institution). ``he endowed the church with lands.''

\textbf{Uphold}: to confirm or support (something that has been questioned); to keep a tradition. ``Thousands of years after the earth was first proved to be a sphere, some groups today uphold the belief that our planet is actually flat.''

\textbf{Notwithstanding}: nevertheless; in spite of this. ``Bad luck notwithstanding , it was still a light day.''

\textbf{Ubiquitous}: present, appearing, or found everywhere. ``his ubiquitous influence was felt by all the family.''

\textbf{Jettison}: throw or drop (something) from an airship. ``Six aircraft jettisoned their loads in the sea.''

\textbf{Propagate}: spread and promote (an idea, theory, etc.) widely. ``the French propagated the idea that the English were violent and gluttonous drunkards.''

\textbf{Abrasive}: (of a substance or material) capable of polishing or cleaning a hard surface by rubbing or grinding. / (of a person or manner) showing little concern for the feelings of others; harsh. ``her abrasive and arrogant personal style won her few friends.''

\textbf{Thaw}: (of ice, snow, or another frozen substance, such as food) become liquid or soft as a result of warming. ``The ice cream was really hard to scoop, so I left it on the counter to thaw for 30 minutes.''

\newpage
\section{Custom Sentences}

``Turkish authorities are myopic and intransigent. Their followers are unlikely to be disaffected soon therefore I cant help but think everything is downhill at this point.''

``I wouldn't say my studying schedule was impeccable. I had allotted a bit too much time on leisure.''

``I consider provincial people to be destitute. It is necessary for them to keep up with the modern times.''

``Sporadic fecund episodes in my life.''

``These days most gestures are acts of mere tokenism, which for those that are aware of could be construed as disingenuousness. Such feelings very well tarnish the once cordial relationship among peers.''

``We all know the saying ''dont make stupid people famous.'' yet they become famous. The people who keep them famous don't make such a quote, nor a quote of opposite meaning. Henceforth, we do not see a quote form their side. 

\end{document}
